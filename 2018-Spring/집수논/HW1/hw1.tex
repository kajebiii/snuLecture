%%%%%%%%%%%%%%%%%%%%%%%%%%%%%%%%%%%%%%%%%
% Programming/Coding Assignment
% LaTeX Template
%
% This template has been downloaded from:
% http://www.latextemplates.com
%
% Original author:
% Ted Pavlic (http://www.tedpavlic.com)
%
% Note:
% The \lipsum[#] commands throughout this template generate dummy text
% to fill the template out. These commands should all be removed when 
% writing assignment content.
%
% This template uses a Perl script as an example snippet of code, most other
% languages are also usable. Configure them in the "CODE INCLUSION 
% CONFIGURATION" section.r
%
%%%%%%%%%%%%%%%%%%%%%%%%%%%%%%%%%%%%%%%%%

%----------------------------------------------------------------------------------------
%	PACKAGES AND OTHER DOCUMENT CONFIGURATIONS
%----------------------------------------------------------------------------------------

\documentclass{article}
\usepackage[hangul]{kotex}
\usepackage{fancyhdr} % Required for custom headers
\usepackage{lastpage} % Required to determine the last page for the footer
\usepackage{extramarks} % Required for headers and footers
\usepackage[usenames,dvipsnames]{color} % Required for custom colors
\usepackage{graphicx} % Required to insert images
\usepackage{listings} % Required for insertion of code
\usepackage{courier} % Required for the courier font
\usepackage{lipsum} % Used for inserting dummy 'Lorem ipsum' text into the template
\usepackage{amsthm,amsmath}
\usepackage[table,xcdraw]{xcolor}

\usepackage{verbatim} % Required for multiple comment
\usepackage{amsmath} % Required for use \therefore \because and others..
\usepackage{amssymb} % Required for use \therefore \because and others..
\usepackage{algorithm, algpseudocode}
\usepackage{verbatim} % for commment, verbatim environment
\usepackage{spverbatim} % automatic linebreak verbatim environment
\usepackage{listings}
\usepackage{ulem}
\usepackage{hyperref}
\usepackage{datetime} % Used for showing version as last modified time
\yyyymmdddate
\DeclareGraphicsExtensions{.pdf,.png,.jpg}

\usepackage{xcolor}




% Margins
\linespread{1.25} % Line spacing
\usepackage[a4paper,top=2cm,bottom=1cm,left=1cm,right=1cm,marginparwidth=1.75cm]{geometry}
\usepackage{titlesec,sectsty}
\setlength\parindent{0pt}
\setlength\parskip{10pt}
\setlength{\abovedisplayskip}{3pt}
\setlength{\belowdisplayskip}{3pt}
\sectionfont{\fontsize{12}{10}\selectfont}
\subsectionfont{\fontsize{10}{10}\selectfont}
\titlespacing{\subsection}{0pt}{-\parskip}{-\parskip}
\usepackage{multicol}
\usepackage{cuted}
\newenvironment{Figure}
{\par\medskip\noindent\minipage{\linewidth}}
{\endminipage\par\medskip}


% mathematics
\usepackage{amsmath,amssymb,mathtools}
\usepackage{cancel} % cancelation of terms
\usepackage{nicefrac} % certain fraction styles
\newcommand*\diff{\mathop{}\!\mathrm{d}} % differential
\DeclareMathOperator{\arsinh}{arsinh}

% Set up the header and footer
\pagestyle{fancy}
\lhead{JongBeom Kim}
\chead{집합과 수리 논리 HW 1} % Top center head
\rhead{2018 Spring} % Top right header
\lfoot{\lastxmark} % Bottom left footer
\cfoot{} % Bottom center footer
\renewcommand\headrulewidth{0.4pt} % Size of the header rule
\renewcommand\footrulewidth{0.4pt} % Size of the footer rule
\newcommand{\myul}[2][black]{\setulcolor{#1}\ul{#2}\setulcolor{black}}

\setlength\parindent{0pt} % Removes all indentation from paragraphs

%----------------------------------------------------------------------------------------
%	CODE INCLUSION CONFIGURATION
%----------------------------------------------------------------------------------------


\setcounter{secnumdepth}{0} % Removes default section numbers
\newcounter{homeworkProblemCounter} % Creates a counter to keep track of the number of problems


\renewcommand\headrule
{
	\begin{minipage}{1\textwidth}
		\hrule width \hsize height 1pt \kern 1.5pt \hrule width \hsize height 0.5pt  
	\end{minipage}\par
}%

% lists
\usepackage{enumitem}
\setlist[itemize]{topsep=-10pt} %set spacing for itemize
\setitemize{itemsep=3pt}
\setlist[enumerate]{topsep=-10pt} %set spacing for itemize
\setenumerate{itemsep=3pt}

%----------------------------------------------------------------------------------------
%	TITLE PAGE
%----------------------------------------------------------------------------------------

%----------------------------------------------------------------------------------------



\begin{document}
\twocolumn



\section{Homework 1}

\subsection{1.1.8.}

어떤 양수 $e > 0$에 대해서는 임의의 자연수 $N$에 대하여 $n \geq N \rightarrow \frac{1}{n} < e$가 성립하지 않는다.

즉, 어떤 양수 $e > 0$에 대해서는, 임의의 자연수 $N$에 대하여, 어떤 $n \geq N$이 존재하여 $\frac{1}{n} \geq e$를 만족한다.

\subsection{1.1.12.}

\begin{align*}
\left(\bigcup_{i \in I} A_i \right) \cap \left(\bigcup_{j \in J} B_j \right) 
=
 \bigcup_{(i, j) \in I \times J} (A_i \cap B_j)
\end{align*}


\subsubsection{1. $\left(\bigcup_{i \in I} A_i \right) \cap \left(\bigcup_{j \in J} B_j \right) \subset \bigcup_{(i, j) \in I \times J} (A_i \cap B_j)$}
\begin{align*}
&x \in \left(\bigcup_{i \in I} A_i \right) \cap \left(\bigcup_{j \in J} B_j \right)
\\ \Rightarrow& \exists a \in I | x \in A_a
\\ & \exists b \in J | x \in B_b
\\ \Rightarrow& x \in A_a \cap B_b
\\ \Rightarrow& x \in \bigcup_{(i, j) \in I \times J} (A_i \cap B_j)
\end{align*}

\subsubsection{2. $\left(\bigcup_{i \in I} A_i \right) \cap \left(\bigcup_{j \in J} B_j \right) \supset \bigcup_{(i, j) \in I \times J} (A_i \cap B_j)$}
\begin{align*}
&x \in \bigcup_{(i, j) \in I \times J} (A_i \cap B_j) 
\\ \Rightarrow& \exists a \in I, \exists b \in J | x \in A_a \cap B_b 
\\ \Rightarrow&x \in A_a, x \in B_b 
\\ \Rightarrow&x \in \left(\bigcup_{i \in I} A_i \right), x \in \left(\bigcup_{j \in J} B_j \right)
\\ \Rightarrow&x \in \left(\bigcup_{i \in I} A_i \right) \cap \left(\bigcup_{j \in J} B_j \right)
\end{align*}

\begin{align*}
\left(\bigcap_{i \in I} A_i \right) \cup \left(\bigcap_{j \in J} B_j \right) 
= 
\bigcap_{(i, j) \in I \times J} (A_i \cup B_j)
\end{align*}

\subsubsection{1. $\left(\bigcap_{i \in I} A_i \right) \cup \left(\bigcap_{j \in J} B_j \right) \subset \bigcap_{(i, j) \in I \times J} (A_i \cup B_j)$}
\begin{align*}
&x \in \left(\bigcap_{i \in I} A_i \right) \cup \left(\bigcap_{j \in J} B_j \right) 
\\ \Rightarrow& x \in \left(\bigcap_{i \in I} A_i \right) \text{또는 }   x \in \left(\bigcap_{j \in J} B_j \right)
\\ \Rightarrow& x \forall i \in I, \forall j \in J | x \in A_i \cup B_j
\\ \Rightarrow& x x \in \bigcap_{(i, j) \in I \times J} (A_i \cup B_j)
\end{align*}


\subsubsection{2. $\left(\bigcap_{i \in I} A_i \right) \cup \left(\bigcap_{j \in J} B_j \right) \supset \bigcap_{(i, j) \in I \times J} (A_i \cup B_j)$}
\begin{align*}
&x \in \bigcap_{(i, j) \in I \times J} (A_i \cup B_j)
\\ \Rightarrow&\forall i \in I, \forall j \in J | x \in A_i \cup B_j
\intertext{만약, }
& x \notin \left(\bigcap_{i \in I} A_i \right)\text{라면}
\\ \Rightarrow& \exists a \in I, x \notin A_a
\\ \Rightarrow& \forall j \in J | x \in A_a \cup B_j
\\ \Rightarrow&x \in \left(\bigcap_{j \in J} B_j \right) \text{이다.}
\intertext{비슷하게} &x \notin \left(\bigcap_{j \in J} B_j \right) 
\Rightarrow x \in \left(\bigcap_{i \in I} A_i \right) 
\\ \therefore &x \in \left(\bigcap_{i \in I} A_i \right) \cup \left(\bigcap_{j \in J} B_j \right) 
\end{align*}


\subsection{1.2.4.}
\subsubsection{(가)}

\begin{align*}
g(f(x_1)) &= g(f(x_2)) 
\intertext{라고 한다면 함수 g가 단사이므로, }
f(x_1) &= f(x_2)  
\intertext{또한 함수 f가 단사이므로, }
x_1 &= x_2
\intertext{따라서, g $\circ$ f는 단사이다.}
\end{align*}
\begin{align*}
f(x_1)) &= f(x_2) 
\intertext{라고 한다면 양번에 함수 g를 취하면, }
(g\circ f)(x_1) &= (g \circ f)(x_2)  
\intertext{함수 g $\circ$ f가 단사이므로, }
x_1 &= x_2
\intertext{따라서 f는 단사함수이다.}
\end{align*}


\subsubsection{(나)}

\begin{align*}
g(f(x)) &=z, (\forall z \in Z)
\intertext{에서 $x \in X$가 존재하는지 확인해보자. 함수 $g$가 전사이므로}
g(y) &= z  
\intertext{인 $y \in Y$가 존재한다. 마찬가지로 함수 $f$가 전사이므로 }
f(x) &= y 
\intertext{인 $x \in X$가 존재한다. 따라서 함수 $g \circ f$는 전사함수이다. }
\end{align*}
\begin{align*}
g(y) &=z, (\forall z \in Z)
\intertext{에서 $y \in Y$가 존재하는지 확인해보자. 함수 $g \circ f$가 전사이므로}
(g\circ f)(x) &= z  
\intertext{인 $x \in X$가 존재한다. 이제 $y$를 }
y &= f(x)
\intertext{로 잡으면 $g(y) = z$을 만족하는 것을 알 수 있다. 따라서 함수 $g$는 전사이다.}
\end{align*}

\subsubsection{(다)}
\begin{align*}
\intertext{$f$와 $g$가 전사이고, 단사이므로 (가), (나)에 의해서 $g \circ f$가 전단사함수임을 알 수 있다.}
\end{align*}
\begin{align*}
&(g \circ f) \circ (f^{-1} \circ g ^{-1})\\
=&g \circ (f \circ f^{-1}) \circ g ^{-1}\\
=&g \circ 1_{Y}\circ g ^{-1}\\
=&g \circ g ^{-1}\\
=&1_{Z}\\
\intertext{또한,}
&(f^{-1} \circ g ^{-1}) \circ (g \circ f)\\
=&f^{-1} \circ (g ^{-1} \circ g) \circ f\\
=&f^{-1} \circ 1_{Y} \circ f\\
=&f^{-1} \circ f\\
=&1_{X}\\
\intertext{따라서 $(g \circ f)^{-1} = f ^ {-1} \circ g ^ {-1}$임을 알 수 있다.}
\end{align*}

\subsection{1.2.11.}
$\exists f : X \rightarrow 2^X$가 전사함수라고 가정하자.

집합 $Y \in 2^X$를 다음과 같이 정의하자. $Y = \left\{ x \in X \mid x \notin f(x) \right\}$

그러면 $f(a) = Y$를 만족하는 $a \in X$가 존재하지 않는다.

\textbf{pf. 귀류법} $\exists a \in X \mid f(a) = Y$라 하자.

그러면 $Y$의 정의에 따라 $a \in f(a)=Y \iff a \notin f(a)=Y$이므로 모순이다. \qed

따라서, $X$에서 $2^X$로 가는 전사함수는 존재하지 않는다.

\subsection{1.2.12.}
\subsubsection{1)}
모든 $A \in 2^X$에 대해 생각해보자.

먼저, $x \in f^{-1}(f(A)) \iff f(x) \in f(A)$ 이다. 

$x \in A \Rightarrow f(x) \in f(A) \iff x \in f^{-1}(f(A))$ 이므로
\begin{align*}
&\therefore f^{-1}(f(A)) \supset A & (1)
\end{align*}
는 항상 성립한다.
\begin{align*}
& x_1, x_2 \in X \mid f(x_1) = f(x_2) \Rightarrow x_1 = x_2
\\ \iff& \left[f(x) \in f(A) \Rightarrow x \in A \right]
\\ \iff& \left[x \in f^{-1}(f(A)) \Rightarrow x \in A \right]
\\ \iff& f^{-1}(f(A)) \subset A & (2)
\end{align*}


\textbf{결론} (1)과 (2)에 의해 함수 $f$가 단사일 필요충분조건은 
\begin{align*}
&f^{-1}(f(A)) = A &A \in 2^X
\end{align*}이다.

\subsubsection{2)}
모든 $B \in 2^Y$에 대해 생각해보자.

먼저, $y \in f(f^{-1}(B)) \iff [\exists x \in f^{-1}(B) \mid f(x) = y] $이다.

\begin{align*}
y \in f(f^{-1}(B)) &\Rightarrow [\exists x \in f^{-1}(B) \mid f(x) = y] 
\\ &\Rightarrow [y = f(x) \in B]
\end{align*}
이므로, 
\begin{align*}
&\therefore f(f^{-1}(B)) \subset B & (1)
\end{align*}
는 항상 성립한다.
\begin{align*}
& \forall y \in Y, \exists x \in X \mid f(x) = y
\\ \iff& \left[y \in B \Rightarrow [\exists x \in f^{-1}(B) \mid f(x) = y]  \right]
\\ \iff& \left[y \in B \Rightarrow y \in f(f^{-1}(B)) \right]
\\ \iff& f(f^{-1}(B)) \supset B & (2)
\end{align*}


\textbf{결론} (1)과 (2)에 의해 함수 $f$가 전사일 필요충분조건은 
\begin{align*}
&f(f^{-1}(B)) = B &B \in 2^Y
\end{align*}이다.

\subsection{1.2.15.}

\begin{align*}f^{-1}\left(\bigcup_{i \in I} B_i\right) = \bigcup_{i \in I} f^{-1}(B_i)\end{align*}
\textbf{pf.}
\begin{align*}
&x \in f^{-1}\left(\bigcup_{i \in I} B_i\right)
\\ \iff& f(x) \in \bigcup_{i \in I} B_i
\\ \iff& \exists i \in I \mid f(x) \in B_i
\\ \iff& \exists i \in I \mid x \in f^{-1}(B_i)
\\ \iff& x \in \bigcup_{i \in I} f^{-1}(B_i)
\end{align*} \qed
\begin{align*}f^{-1}\left(\bigcap_{i \in I} B_i\right) = \bigcap_{i \in I} f^{-1}(B_i)\end{align*}
\textbf{pf.}
\begin{align*}
&x \in f^{-1}\left(\bigcap_{i \in I} B_i\right)
\\ \iff& f(x) \in \bigcap_{i \in I} B_i
\\ \iff& \forall i \in I \mid f(x) \in B_i
\\ \iff& \forall i \in I \mid x \in f^{-1}(B_i)
\\ \iff& x \in \bigcap_{i \in I} f^{-1}(B_i) 
\end{align*} \qed

\begin{align*}f\left(\bigcup_{i \in I} A_i\right) = \bigcup_{i \in I} f(A_i)\end{align*}
\textbf{pf.}
\begin{align*}
&y \in f\left(\bigcup_{i \in I} A_i\right)
\\ \iff& \exists x \in \bigcup_{i \in I} A_i \mid f(x) = y
\\ \iff& \exists i \in I \mid \left[\exists x \in A_i \mid f(x) = y\right]
\\ \iff& \bigcup_{i \in I} f(A_i)
\end{align*} \qed

\begin{align*}&f\left(\bigcap_{i \in I} A_i\right) \subset \bigcap_{i \in I} f(A_i) &(1)\end{align*}
\textbf{pf.} 네번째 줄로 넘어갈 때 $\iff$가 아닌 $\Rightarrow$임에 유의하라.
\begin{align*}
&y \in f\left(\bigcap_{i \in I} A_i\right)
\\ \iff& \exists x \in \bigcap_{i \in I} A_i \mid f(x) = y
\\ \iff& \left[\exists x \mid \forall i\in I, x \in A_i  \right] \mid  f(x) = y 
\\ \Rightarrow& \forall i \in I \mid \left[\exists x \in A_i \mid f(x) = y\right]
\\ \iff& \bigcap_{i \in I} f(A_i)
\end{align*} \qed

\begin{align*}f\left(\bigcap_{i \in I} A_i\right) = \bigcap_{i \in I} f(A_i)\end{align*}
의 필요충분조건을 찾는 것은
\begin{align*}f\left(\bigcap_{i \in I} A_i\right) \supset \bigcap_{i \in I} f(A_i)\end{align*}
의 필요충분조건을 찾는 것과 같다. ($\therefore$ 식 (1))
\begin{align*}
&f\left(\bigcap_{i \in I} A_i\right) \supset \bigcap_{i \in I} f(A_i)
\\ \iff & \forall i \in I \mid \left[\exists x \in A_i \mid f(x) = y\right] 
\\ &\Rightarrow \left[\exists x \mid \forall i\in I, x \in A_i  \right] \mid  f(x) = y  
\\ \iff & \left[ f(x_1) = f(x_2) \iff x_1 = x_2 \right]
\end{align*}
이므로, 
\begin{align*}
f\left(\bigcap_{i \in I} A_i\right) = \bigcap_{i \in I} f(A_i) \iff \left[ f(x_1) = f(x_2) \iff x_1 = x_2 \right]
\end{align*}
즉, 함수 $f$가 단사인 것과 필요충분조건이다.

\subsection{1.3.1.}
\subsubsection{(1) $\wedge$ (2) $\wedge$ $\lnot$(3)}
$R = \left\{(a, a), (b, b), (c, c), (a, b), (b, a), (b, c), (c, b) \right\}$
\\$(a, b) \in R$ $(b, c) \in R$이지만 $(a, c) \notin R$이다.

\subsubsection{$\lnot$(1) $\wedge$ (2) $\wedge$ (3)}
$R = \left\{(c, c) \right\}$
\\$(a, a) \notin R$이다.

\subsubsection{(1) $\wedge$ $\lnot$(2) $\wedge$ (3)}
$R = \left\{(a, a), (b, b), (c, c), (a, b), (b, c), (a, c) \right\}$
\\$(a, b) \in$이지만 $(b, a) \notin R$이다.

\subsection{1.3.2.}
\begin{align*}
\exists (x, y) \in R \Rightarrow (y, x) \in R
\\ (x, y) \in R, (y, x) \in R \Rightarrow (x, x) \in R
\end{align*}
주장은 위와 같다. 하지만 결국 $(x, x) \in R$이 만족하기 위해서는 $(x, y) \in R$이어야 한다.
즉, $R$에서 $(x, y)$꼴의 원소가 없다면 $(x, x)$가 $R$의 원소일 보장은 없다.

예를 들어, 원소 세 개인 집합 $X = \left\{a, b, c \right\}$에서 $R = \left\{ (c, c) \right\}$인 경우가 있다.
 
\subsection{1.3.7.}
\begin{align*}
k \in \mathbb{N}, m \sim n \iff k \mid m - n
\end{align*}
\subsubsection{1)}
\begin{align*}
&\forall x \in \mathbb{Z}, k \mid x - x
\\ \Rightarrow& \forall x \in \mathbb{Z}, x \sim x
\end{align*}

\subsubsection{2)}
\begin{align*}
x \sim y \Rightarrow& k \mid x - y
\\ \Rightarrow& k \mid -(x - y)
\\ \Rightarrow& k \mid y - x
\\ \Rightarrow& y \sim x
\end{align*}

\subsubsection{3)}
\begin{align*}
x \sim y, y \sim z \Rightarrow& k \mid x - y, k \mid y - z
\\ \Rightarrow& k \mid (x - y) + (y - z)
\\ \Rightarrow& k \mid x - z
\\ \Rightarrow& x \sim z
\end{align*}
\textbf{결론} 따라서 $\sim$은 동치관계이다.
\begin{align*}
\mathbb{Z} / \sim =& \left\{\left[m \right] \mid m \in \mathbb{Z} \right\}
\\ =& \left\{\left[0 \right], \left[1 \right], \cdots, \left[k-1 \right] \right\}
\end{align*}

\subsection{1.3.12.}
\subsubsection{1)}
$x, y \in V/W \Rightarrow x + y \in V/W$이고, 
\\$x \in V/W, a \in \mathbb{R} \Rightarrow ax \in V/W$ 이므로
$V/W$는 벡터공간이다.

\subsubsection{2)}
$\phi, \widetilde{\phi}$는 선형사상이다.

$\phi : V \rightarrow Z$에 대하여 $\widetilde{\phi} \circ q = \phi$를 만족하는 $\widetilde{\phi} : V/W \rightarrow Z$가 존재 할 필요충분조건은
\begin{align*}
x \sim_{W} y \Rightarrow \phi(x) = \phi(y)
\end{align*}
이다. 한편,
\begin{align*}
&x \sim_{W} y \iff x - y \in W & (1)
\end{align*}
\begin{align*}
\phi(x) = \phi(y) \iff& \phi(x) - \phi(y) = 0 &
\\ \iff& \phi(x - y) = 0 &\text{($\because \phi$는 선형사상)}
\\ \therefore \phi(x) = \phi(y) \iff& \phi(x - y) = 0& (2)
\end{align*}
인 것을 확인할 수 있다. 이를 이용하면, 
\begin{align*}
&\left[ x \sim_{W} y \Rightarrow \phi(x) = \phi(y) \right] &
\\ \iff &\left[ x - y \in W \Rightarrow \phi(x - y) = 0 \right] &(\because (1), (2))
\\ \iff &\left[ x \in W \Rightarrow \phi(x) = 0 \right] &
\end{align*}
따라서, $ x \in W \Rightarrow \phi(x) = 0$은 필요충분조건이다.

\subsection{1.4.1.}
집합 $X = \left\{a, b, c\right\}$에 대해 관계 $R \subset X \times X$를 생각하자.
\subsubsection{(1) $\wedge$ (2) $\wedge$ $\lnot$(3)}
$R = \left\{(a, a), (b, b), (c, c), (a, b), (b, c) \right\}$
\\$(a, b) \in R$ $(b, c) \in R$이지만 $(a, c) \notin R$이다.

\subsubsection{$\lnot$(1) $\wedge$ (2) $\wedge$ (3)}
$R = \left\{(c, c) \right\}$
\\$(a, a) \notin R$이다.

\subsubsection{(1) $\wedge$ $\lnot$(2) $\wedge$ (3)}
$R = \left\{ (a, a), (b, b), (c, c), (a, b), (b, a) \right\}$
\\$(a, b) \in R$ $(b, a) \in R$이지만 $a \neq b$이다.



\subsection{1.4.5.}
\begin{align*}
\forall A \in \mathbb{A} \Rightarrow \bigcap \mathbb{A} \subset A
\end{align*}
이므로, $\bigcap \mathbb{A}$는 $\mathbb{A}$의 하계이다.
\\ 만약 $I \in 2^{X}$가 $\mathbb{A}$의 하계라면, $\forall A \in \mathbb{A}, I \subset A$이므로,
\begin{align*}
x \in I \Rightarrow& \forall A \in \mathbb{A}, x \in A
\\ \Rightarrow& x \in \bigcap \mathbb{A}
\\ \therefore I \subset& \bigcap \mathbb{A}
\end{align*}
따라서, $\text{inf} \mathbb{A} = \bigcap \mathbb{A}$


\subsection{1.4.7.}
\begin{align*}
\forall c \in \mathbb{F}(C), c \subset C
\end{align*}이고,
\begin{align*}
&\exists S \in \mathbb{F}(C) \mid \forall c \in \mathbb{F}(C), c \subset S
\\ \iff& C \subset S
\end{align*}이므로,
\begin{align*}
\text{sup} \mathbb{F}(C) = C
\end{align*}

\begin{align*}
\forall c \in \mathbb{F}(C), c \supset \varnothing
\end{align*}이고,
\begin{align*}
&\exists I \in \mathbb{F}(C) \mid \forall c \in \mathbb{F}(C), c \supset I
\\ \iff&  \varnothing \supset I
\end{align*}이므로,
\begin{align*}
\text{inf} \mathbb{F}(C) = \varnothing
\end{align*}

\subsection{1.4.9.}
\begin{align*}
F \vee G = \text{sup}\left\{F, G\right\} =  F \cup G
\\ F \wedge G = \text{inf}\left\{F, G\right\}  = F \cap G
\end{align*}

\end{document}










































