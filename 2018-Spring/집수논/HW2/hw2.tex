%%%%%%%%%%%%%%%%%%%%%%%%%%%%%%%%%%%%%%%%%
% Programming/Coding Assignment
% LaTeX Template
%
% This template has been downloaded from:
% http://www.latextemplates.com
%
% Original author:
% Ted Pavlic (http://www.tedpavlic.com)
%
% Note:
% The \lipsum[#] commands throughout this template generate dummy text
% to fill the template out. These commands should all be removed when 
% writing assignment content.
%
% This template uses a Perl script as an example snippet of code, most other
% languages are also usable. Configure them in the "CODE INCLUSION 
% CONFIGURATION" section.r
%
%%%%%%%%%%%%%%%%%%%%%%%%%%%%%%%%%%%%%%%%%

%----------------------------------------------------------------------------------------
%	PACKAGES AND OTHER DOCUMENT CONFIGURATIONS
%----------------------------------------------------------------------------------------

\documentclass{article}
\usepackage[hangul]{kotex}
\usepackage{fancyhdr} % Required for custom headers
\usepackage{lastpage} % Required to determine the last page for the footer
\usepackage{extramarks} % Required for headers and footers
\usepackage[usenames,dvipsnames]{color} % Required for custom colors
\usepackage{graphicx} % Required to insert images
\usepackage{listings} % Required for insertion of code
\usepackage{courier} % Required for the courier font
\usepackage{lipsum} % Used for inserting dummy 'Lorem ipsum' text into the template
\usepackage{amsthm,amsmath}
\usepackage[table,xcdraw]{xcolor}

\usepackage{verbatim} % Required for multiple comment
\usepackage{amsmath} % Required for use \therefore \because and others..
\usepackage{amssymb} % Required for use \therefore \because and others..
\usepackage{algorithm, algpseudocode}
\usepackage{verbatim} % for commment, verbatim environment
\usepackage{spverbatim} % automatic linebreak verbatim environment
\usepackage{listings}
\usepackage{ulem}
\usepackage{hyperref}
\usepackage{datetime} % Used for showing version as last modified time
\yyyymmdddate
\DeclareGraphicsExtensions{.pdf,.png,.jpg}

\usepackage{xcolor}




% Margins
\linespread{1.25} % Line spacing
\usepackage[a4paper,top=2cm,bottom=1cm,left=1cm,right=1cm,marginparwidth=1.75cm]{geometry}
\usepackage{titlesec,sectsty}
\setlength\parindent{0pt}
\setlength\parskip{10pt}
\setlength{\abovedisplayskip}{3pt}
\setlength{\belowdisplayskip}{3pt}
\sectionfont{\fontsize{12}{10}\selectfont}
\subsectionfont{\fontsize{10}{10}\selectfont}
\titlespacing{\subsection}{0pt}{-\parskip}{-\parskip}
\usepackage{multicol}
\usepackage{cuted}
\newenvironment{Figure}
{\par\medskip\noindent\minipage{\linewidth}}
{\endminipage\par\medskip}


% mathematics
\usepackage{amsmath,amssymb,mathtools}
\usepackage{cancel} % cancelation of terms
\usepackage{nicefrac} % certain fraction styles
\newcommand*\diff{\mathop{}\!\mathrm{d}} % differential
\DeclareMathOperator{\arsinh}{arsinh}

% Set up the header and footer
\pagestyle{fancy}
\lhead{JongBeom Kim}
\chead{집합과 수리 논리 HW 2} % Top center head
\rhead{2018 Spring} % Top right header
\lfoot{\lastxmark} % Bottom left footer
\cfoot{} % Bottom center footer
\renewcommand\headrulewidth{0.4pt} % Size of the header rule
\renewcommand\footrulewidth{0.4pt} % Size of the footer rule
\newcommand{\myul}[2][black]{\setulcolor{#1}\ul{#2}\setulcolor{black}}


\setlength\parindent{0pt} % Removes all indentation from paragraphs

%----------------------------------------------------------------------------------------
%	CODE INCLUSION CONFIGURATION
%----------------------------------------------------------------------------------------


\setcounter{secnumdepth}{0} % Removes default section numbers
\newcounter{homeworkProblemCounter} % Creates a counter to keep track of the number of problems


\renewcommand\headrule
{
	\begin{minipage}{1\textwidth}
		\hrule width \hsize height 1pt \kern 1.5pt \hrule width \hsize height 0.5pt  
	\end{minipage}\par
}%

% lists
\usepackage{enumitem}
\setlist[itemize]{topsep=-10pt} %set spacing for itemize
\setitemize{itemsep=3pt}
\setlist[enumerate]{topsep=-10pt} %set spacing for itemize
\setenumerate{itemsep=3pt}

%----------------------------------------------------------------------------------------
%	TITLE PAGE
%----------------------------------------------------------------------------------------

%----------------------------------------------------------------------------------------



\begin{document}
\twocolumn



\chapter{Homework 2}

\section{2.1.2.}
\subsection{(가)}
\begin{align*}
0+n = \gamma_0(n) = n
\end{align*}
\textbf{pf.} $X = \left\{ n \in \mathbb{N} \mid \gamma_0(n) = n \right\}$라 하자.
\begin{align*}
\gamma_0(0) = 0
\end{align*}이므로 $0 \in X$이다. 이제 $n \in X$라고 가정하면,
\begin{align*}
\gamma_0(n^+) = [\gamma_0(n)]^+ = [n]^+ = n^+
\end{align*}이므로 $n^+ \in X$이다. 따라서 수학적 귀납법에 의해 $X = \mathbb{N}$이고, 임의의 자연수 $n \in \mathbb{N}$에 대해
\begin{align*}
0 + n = n
\end{align*}가 성립한다.

\begin{align*}
(m+n)+k = \gamma_{\gamma_{m}(n)}(k) = \gamma_{m}(\gamma_{n}(k))=m+(n+k)
\end{align*}
\textbf{pf.} $m, n$을 고정하고 $k \in \mathbb{N}$에 대해 증명해도 충분하다. 
\\$X = \left\{k\in \mathbb{N} \mid (m+n)+k = m+(n+k) \right\}$라 하자.
\begin{align*}
(m+n)+0 = \gamma_{m+n}(0) = m+n
\\ m+(n+0) = \gamma_m(\gamma_n(0)) = \gamma_m(n) = m+n
\end{align*}이므로 $0 \in X$이다. 이제 $k \in X$라고 가정하면,
\begin{align*}
(m+n)+k^+ =& \gamma_{m+n}(k^+) = [\gamma_{m+n}(k)]^+ = [(m+n)+k]^+
\\=&[m+(n+k)]^+ = [\gamma_m(\gamma_n(k))]^+ 
\\ =& \gamma_m([\gamma_n(k)]^+) = \gamma_{m}(\gamma_n(k^+))  = m+(n+k^+) 
\end{align*}이므로 $k^+ \in X$이다. 따라서 수학적 귀납법에 의해 $X = \mathbb{N}$이고, 임의의 자연수 $k \in \mathbb{N}$에 대해
\begin{align*}
(m+n)+k = m+(n+k)
\end{align*}가 성립한다. 


\begin{align*}
m+n = \gamma_m(n) = \gamma_n(m) = n+m
\end{align*}
\textbf{pf.} $m$을 고정하고 $n \in \mathbb{N}$에 대해 증명해도 충분하다. 
\\$X = \left\{n \in \mathbb{N} \mid m+n=n+m \right\}$라 하자.
\begin{align*}
m+0 = \gamma_m(0) = m = \gamma_0(m) = 0+m
\end{align*}이므로 $0 \in X$이다. 이제 $n \in X$라고 가정하면,
\begin{align*}
m+n^+ =& \gamma_m(n^+) = [\gamma_m(n)]^+ = [\gamma_n(m)]^+ = \gamma_n(m^+)
\\ =& \gamma_n(1+m) = n + (1+m) = (n+1)+m 
\\ =& (n+0)^+ + m= n^+ + m
\end{align*}이므로 $n^+ \in X$이다. 따라서 수학적 귀납법에 의해 $X = \mathbb{N}$이고, 임의의 자연수 $n \in \mathbb{N}$에 대해
\begin{align*}
m+n = n+m
\end{align*}가 성립한다. 

\subsection{(나)}
\begin{align*}
0n = \delta_0(n) = 0
\end{align*}
\textbf{pf.} $X = \left\{ n \in \mathbb{N} \mid \delta_0(n) = 0 \right\}$라 하자.
\begin{align*}
\delta_0(0) = 0
\end{align*}이므로 $0 \in X$이다. 이제 $n \in X$라고 가정하면,
\begin{align*}
\delta_0(n^+) = \delta_0(n)+0 = \delta_0(n) = 0n = 0
\end{align*}이므로 $n^+ \in X$이다. 따라서 수학적 귀납법에 의해 $X = \mathbb{N}$이고, 임의의 자연수 $n \in \mathbb{N}$에 대해
\begin{align*}
0n = 0
\end{align*}가 성립한다.

\begin{align*}
1n = \delta_1(n) = 1
\end{align*}
\textbf{pf.} $X = \left\{ n \in \mathbb{N} \mid \delta_1(n) = n \right\}$라 하자.
\begin{align*}
\delta_1(0) = 0
\end{align*}이므로 $0 \in X$이다. 이제 $n \in X$라고 가정하면,
\begin{align*}
\delta_1(n^+) = \delta_1(n)+1 = n+1 = n^+
\end{align*}이므로 $n^+ \in X$이다. 따라서 수학적 귀납법에 의해 $X = \mathbb{N}$이고, 임의의 자연수 $n \in \mathbb{N}$에 대해
\begin{align*}
1n = n
\end{align*}가 성립한다.


\textbf{Lemma 1.} $m(n+k) = mn + mk$이다.
\textbf{pf.} $m, n$을 고정하고 $k \in \mathbb{N}$에 대해 증명해도 충분하다. 
\\$X = \left\{k\in \mathbb{N} \mid m(n+k) = mn+mk) \right\}$라 하자.
\begin{align*}
m(n+0) = mn
\\ mn+m0 = mn+0 = mn
\end{align*}이므로 $0 \in X$이다. 이제 $k \in X$라고 가정하면,
\begin{align*}
m(n+k^+) =& m[(n+k)^+] = m(n+k) + m
\\ =& mn + mk + m = = mn + m(k^+)
\end{align*}이므로 $k^+ \in X$이다. 따라서 수학적 귀납법에 의해 $X = \mathbb{N}$이고, 임의의 자연수 $k \in \mathbb{N}$에 대해
\begin{align*}
m(n+k) = mn+mk
\end{align*}가 성립한다. 

\begin{align*}
(mn)k = \delta_{\delta_{m}(n)}(k) = \delta_{m}(\delta_{n}(k))=m(nk)
\end{align*}
\textbf{pf.} $m, n$을 고정하고 $k \in \mathbb{N}$에 대해 증명해도 충분하다. 
\\$X = \left\{k\in \mathbb{N} \mid (mn)k = m(nk) \right\}$라 하자.
\begin{align*}
(mn)0 = \delta_{mn}(0) = 0
\\ m(n0) = \delta_m(\delta_n(0)) = \delta_m(0) = 0
\end{align*}이므로 $0 \in X$이다. 이제 $k \in X$라고 가정하면,
\begin{align*}
(mn)k^+ =& mn(k^+) = mn(k) + mn = m(nk) + mn 
\\ =& m(nk) + m(n) = m(nk + n) \tag{$\because$ Lemma 1.}
\\ =&  m(nk + (0+n)) = m(nk + n(0^+))  
\\ =& m(n(k+0^+)) = m[n(k^+)]
\end{align*}이므로 $k^+ \in X$이다. 따라서 수학적 귀납법에 의해 $X = \mathbb{N}$이고, 임의의 자연수 $k \in \mathbb{N}$에 대해
\begin{align*}
(mn)k = m(nk)
\end{align*}가 성립한다. 

\textbf{Lemma 2.} $(m^+)n = mn + n$이다.
\\ \textbf{pf.} $m$을 고정하고 $n \in \mathbb{N}$에 대해 증명해도 충분하다. 
\\$X = \left\{n \in \mathbb{N} \mid (m^+)n=mn + n \right\}$라 하자.
\begin{align*}
(m^+)(0) = 0 = m0 + 0
\end{align*}이므로 $0 \in X$이다. 이제 $n \in X$라고 가정하면,
\begin{align*}
(m^+)(n^+) =& (m^+)(n) + m^+ = mn + n + m^+
\\ =& mn + n + m + 1 = mn + m + n + 1
\\ =& m(n^+) + n^+
\end{align*}이므로 $n^+ \in X$이다. 따라서 수학적 귀납법에 의해 $X = \mathbb{N}$이고, 임의의 자연수 $n \in \mathbb{N}$에 대해
\begin{align*}
(m^+)n = mn + n
\end{align*}가 성립한다. 



\begin{align*}
mn = \delta_m(n) = \delta_n(m) = nm
\end{align*}
\textbf{pf.} $m$을 고정하고 $n \in \mathbb{N}$에 대해 증명해도 충분하다. 
\\$X = \left\{n \in \mathbb{N} \mid mn=nm \right\}$라 하자.
\begin{align*}
m0 = \delta_m(0) = 0 = \delta_0(m) = 0m
\end{align*}이므로 $0 \in X$이다. 이제 $n \in X$라고 가정하면,
\begin{align*}
m(n^+) =& mn + m = nm + m = nm + m 
\\=& (n^+)m \tag{$\because$ Lemma 2.}
\end{align*}이므로 $n^+ \in X$이다. 따라서 수학적 귀납법에 의해 $X = \mathbb{N}$이고, 임의의 자연수 $n \in \mathbb{N}$에 대해
\begin{align*}
m+n = n+m
\end{align*}가 성립한다. 


\subsection{(다)}
\begin{align*}
m(n+k) = mn + mk
\end{align*}
\textbf{pf.} Lemma 1.과 같다. 위를 참고하라.

\begin{align*}
(n+k)m = nm + km
\end{align*}
\textbf{pf.} 지금까지 증명한 것을 이용하여 보일 수 있다.
\begin{align*}
(n+k)m = m(n+k) = mn + mk = nm + km
\end{align*}이므로 성립한다.

\section{2.1.6.}
\begin{align*}
n \in \mathbb{N} \implies n = 0 \text{ or } [\exists m \in \mathbb{N} \mid n = m^+]
\end{align*}
\textbf{pf.} $X = \left\{ n \in \mathbb{N} \mid n = 0 \text{ or } [\exists m \in \mathbb{N} \mid n = m^+] \right\}$라 하자.
\begin{align*}
0 = 0
\end{align*}이므로 $0 \in X$이다. 이제 $n \in X$라고 가정할 때,
\begin{align*}
\exists m = n \in \mathbb{N} \mid n^+ = (m)^+
\end{align*}은 가정에 관계없이 참이므로 $n^+ \in X$이다. 따라서 수학적 귀납법에 의해 $X = \mathbb{N}$이고, 임의의 자연수 $n \in \mathbb{N}$에 대해
\begin{align*}
n = 0 \text{ or } [\exists m \in \mathbb{N} \mid n = m^+] 
\end{align*}가 성립한다.


\section{2.1.7.}
\textbf{Lemma 1.} $m < n \iff m^+ \le n$이다.
\\\textbf{pf.} ($\implies$)는 책의 (2.6)이다.
\\($\impliedby$)
\\ $m^+ \le n \iff m^+ = n \text{ or } m^+ \in n$이고, $m < n \iff m \in n$임을 상기하자.
\begin{align*}
&m^+ = n \implies n = m \cup \left\{m \right\} \implies m \in n \implies m < n
\\&m^+ \in n \implies m^+ < n \implies m < m^+ < n \implies m < n
\end{align*}이므로 성립한다. \qed

\textbf{Lemma 2.} $m \le n \iff m^+ \le n^+$임을 보이자.
\begin{align*}
m^+ \in n^+ \iff& m^+ \in n \cup \left\{n \right\}
\\ \iff& m^+ = n \text{ or } m^+ \in n
\\ \iff& m^+ \le n
\\ \iff& m < n  \tag{$\because$ Lemma 1.}
\\ \iff& m \in n
\end{align*}이고, 
\begin{align*}
m = n \iff m^+ = n^+
\end{align*}이므로
\begin{align*}
[m = n \text{ or } m \in n] \iff [m^+ = n^+ \text{ or } m^+ \in n^+] 
\\ m \le n \iff m^+ \le n^+
\end{align*}이다. \qed

\subsection{(가)}
\begin{align*}
m + k \le m + l \iff k \le l
\end{align*}
\textbf{pf.} $X = \left\{ m \in \mathbb{N} \mid m + k \le m + l \iff k \le l
 \right\}$라 하자.
\begin{align*}
k = 0 + k \le 0 + l = l \iff k \le l
\end{align*}이므로 $0 \in X$이다. 이제 $m \in X$라고 가정할 때,
\begin{align*}
m^+ + k \le m^+ + l \iff& m + 1 + k \le m + 1 + l
\\ \iff& m + k + 1 \le m + l + 1
\\ \iff& (m+k)^+ \le (m+l)^+
\\ \iff& m+k \le m+l \tag{$\because$ Lemma 2.}
\\ \iff& k \le l
\end{align*}이므로 $m^+ \in X$이다. 따라서 수학적 귀납법에 의해 $X = \mathbb{N}$이고, 임의의 자연수 $m \in \mathbb{N}$에 대해
\begin{align*}
m + k \le m + l \iff k \le l
\end{align*}가 성립한다.

\subsection{(나)}
\begin{align*}
mk \le ml \iff k \le l \tag{단, $m \neq 0$}
\end{align*}
\textbf{pf.} $X = \left\{ k \in \mathbb{N} \mid m \neq 0,  mk \le ml \iff k \le l \right\}$라 하자.
\begin{align*}
0=0k \leq ml \iff 0=k \leq l
\end{align*}는 모든 자연수가 $0$보다 크거나 같으므로 성립한다. 

따라서, $0 \in X$이다. 이제 $k \in X$라고 가정할 때 $k^+ \in X$임을 보이자. 일단 $l \in \mathbb{N}$이므로 $\l = 0 \text{ or } l = p^+$이다.

$l = 0$인 경우를 먼저 생각하자. 
\begin{align*}
n \le 0 \iff n = 0 \text{ or } n \in 0 = \varnothing \iff n = 0
\end{align*}인 것을 알고있다. $m \neq 0$인 것을 참고하여,
\begin{align*}
mk \le m0 = 0 \iff& mk = 0
\\ \iff& k = 0
\\ \iff& k \le 0
\end{align*}이므로, $l=0$인 경우에는 $mk \le ml \iff k \le l$가 성립한다.

$l = p^+$인 경우를 생각해보자.
\begin{align*}
m(k^+) \le m(p^+) \iff& m(k+1) \le m(p+1)
\\ \iff& mk + m \le mp + m
\\ \iff& m + mk \le m + mp
\\ \iff& mk \le mp \tag{$\because$ (가)}
\\ \iff& k \le p
\\ \iff& k^+ \le p^+ \tag{$\because$ Lemma 2.}
\\ \iff& k^+ \le l
\end{align*}이므로 $k^+ \in X$이다. 따라서 수학적 귀납법에 의해 $X = \mathbb{N}$이고, 임의의 자연수 $k \in \mathbb{N}$에 대해
\begin{align*}
m \neq 0 \text{인 경우,  }  mk \le ml \iff k \le l
\end{align*}가 성립한다.


\section{2.2.4.}
\subsection{Well-defined}
\begin{align*}
[m, n] + [k, l] = [m+k, n+l]
\end{align*}이 잘 정의되었는지 확인하자.
\\$[m, n] = [m', n']$이고 $[k, l] = [k', l']$일 때
\begin{align*}
(m+k)+(n'+l') =& (m'+k')+(n'+l') 
\\=& (m'+k')+(n+l) 
\end{align*}이므로 $[m+n, k+l] = [m'+n', k'+l']$이다. 따라서,
\begin{align*}
[m, n]+[k, l] = [m', n'] + [k', l']
\end{align*}이다. 따라서, (2. 12)의 연산은 잘 정의되었다.

\subsection{결합법칙}
\begin{align*}
([a, b] + [c, d]) + [e, f] =& [a+c, b+d] + [e, f]
\\ =& [a+c+e, b+d+f]
\\ =& [a, b] + [c+e, d+f]
\\ =& [a, b] + ([c, d] + [e, f])
\end{align*}이므로 결합법칙이 성립한다.

\subsection{교환법칙}
\begin{align*}
[a, b] + [c, d] =& [a+c, b+d]
\\ =& [c+a, d+b]
\\ =& [c, d] + [a, b]
\end{align*}이므로 교환법칙이 성립한다.


\section{2.2.6.}
\begin{align*}
[m, n] \cdot [k, l] = [0, 0] \iff& [mk + nl, ml + nk] = [0, 0]
\\ \iff& mk+nl+0 = ml+nk+0
\\ \iff& mk+nl = ml+nk
\end{align*}
여기서 $m = n$이면 $[m, n] = [0, 0]$인 경우이고,
\begin{align*}
mk+nl =& nk + nl
\\ =& nk + ml
\\ =& ml + nk
\end{align*}이므로, $[m, n] \cdot [k, l] = [0, 0]$가 성립한다.

이제, $m \neq n$인 경우를 생각하자.
\\ \textbf{1. $m < n$인 경우}
\\ $n = m + p$를 만족하는 $p \in \mathbb{N}$이 존재한다. 단, $m \neq n$이므로 $p \neq 0$이다.
\begin{align*}
mk+nl = ml+nk \iff& mk+(m+p)l = ml + (m+p)k
\\ \iff& mk+ml+pl = ml+mk+pk
\\ \iff& (mk+ml)+pl = (mk+ml)+pk
\\ \iff& pl = pk \iff pl \le pk \text{ and } pk \le pl
\\ \iff& l \le k \text{ and } k \le l \iff k = l
\\ \iff& [k, l] = [0, 0]
\end{align*}이므로, $[k, l] = [0, 0]$이다.

\textbf{2. $n < m$인 경우}
\\ 1.와 비슷하게 진행하면 된다. $m = n + p$를 만족하는 $p \in \mathbb{N}$이 존재한다. 단, $m \neq n$이므로 $p \neq 0$이다.
\begin{align*}
mk+nl = ml+nk \iff& (n+p)k+nl = (n+p)l + nk
\\ \iff& nk+pk+nl = nl+pl+nk
\\ \iff& (nk+nl)+pk = (nk+nl)+pl
\\ \iff& pk = pl \iff pl \le pk \text{ and } pk \le pl
\\ \iff& l \le k \text{ and } k \le l \iff k = l
\\ \iff& [k, l] = [0, 0]
\end{align*}이므로, $[k, l] = [0, 0]$이다.

\textbf{결론} 
\\$[m, n] \cdot [k, l] = [0, 0]$이면, $[m, n] = [0, 0] \text{ or } [k, l] = [0, 0]$이다.

\section{2.2.8.}
$a = [x, 0]\in P$이고, $b = [y, 0] \in P$라 하자.  가정에 의해, $x, y \in \mathbb{N}$이고, $x \neq 0, y \neq 0$이다.
\begin{align*}
x \neq 0 \text{ and } y \neq 0 \iff& 0 < x \text{ and } 0 < y
\\ \iff& 0 < x \text{ and } x < x + y
\\ \iff& 0 < x < x + y
\\ \iff& 0 < x + y
\end{align*}이다. 그러면,
\begin{align*}
a+b = [x, 0] + [y, 0] = [x+y, 0+0] = [x+y, 0]
\end{align*}이고, $x+y \in \mathbb{N}, 0 < x+y$이므로, $a+b \in P$이다.

\begin{align*}
x \neq 0 \text{ and } y \neq 0 \iff& 0 < x \text{ and } 0 < y
\\ \iff& 1 \le x \text{ and } 0 < y
\\ \iff& y \le xy \text{ and } 1 \le y
\\ \iff& 1 \le xy \iff 0 < xy
\end{align*}이다. 그러면,
\begin{align*}
ab = [x, 0] \cdot [y, 0] = [xy+0, 0+0] = [xy, 0]
\end{align*}이고, $xy \in \mathbb{N}, 0 < xy$이므로, $ab \in P$이다.


\section{2.2.15.}
임의의 $[a, b], [c, d] \in \mathbb{Q}$에 대하여
\begin{align*}
[a, b] \ge [c, d] \iff& [a, b] - [c, d] \in P_{\mathbb{Q}} \text{ or } [a, b] = [c, d]
\end{align*}이다. 여기서 $[a, b] = [c, d]$의 동치관계는
\begin{align*}
[a, b] = [c, d] \iff& ad = cb
\\ \iff& ad(bd) = cb(bd) \tag{$\because bd \neq 0$}
\\ \iff& abd^2 = cdb^2
\end{align*}인 것을 알 수 있다. 다음으로 $[a, b] - [c, d] \in P_{\mathbb{Q}}$의 동치관계는
\begin{align*}
[a, b] - [c, d] \in P_{\mathbb{Q}} \iff& [a, b] + (-[c, d]) \in P_{\mathbb{Q}}
\\ \iff& [a, b] + [-c, d] \in P_{\mathbb{Q}}
\\ \iff& [ad+(-c)b, bd] \in P_{\mathbb{Q}}
\\ \iff& [ad-cb, bd] \in P_{\mathbb{Q}}
\\ \iff& (ad-cb, bd) \in P_{\mathbb{Z}} \times P_{\mathbb{Z}}
\\ \iff& ad-cb \in P_{\mathbb{Z}} \text{ and } bd \in P_{\mathbb{Z}}
\\ \iff& 0 < ad-cb \text{ and } 0 < bd
\\ \iff& ad < cb \text{ and } 0 < bd
\\ \iff& ad(bd) < cb(bd)
\\ \iff& abd^2 < cdb^2
\end{align*}이다. 따라서, 
\begin{align*}
[a, b] \ge [c, d] \iff& [a, b] - [c, d] \in P_{\mathbb{Q}} \text{ or } [a, b] = [c, d]
\\ \iff& abd^2 < cdb^2 \text{ or } abd^2 = cdb^2
\\ \iff& abd^2 = cdb^2
\end{align*}이다. 

\section{2.3.1.}
임의의 
\begin{align*}
r = \frac{a}{b} \in \mathbb{Q} \tag{$a \in \mathbb{Z}, b \in P_{\mathbb{Z}} $}
\end{align*}
에 대해
\begin{align*}
r^* = \left\{ p \in \mathbb{Q} \mid p < r \right\}
\end{align*}가 절단임을 보이자.

\textbf{(절 1)} $r-1 < r$이므로 $r-1 \in r^*$이고, $r+1 \ge r$이므로 $r+1 \notin r^*$이다.
따라서 $r^* \neq \varnothing$, $r^* \neq \mathbb{Q}$이다.

\textbf{(절 2)} $p \in r^*, q \in \mathbb{Q}, q < p$라고 하면,
\begin{align*}
p < r \text{ and } q < p \iff& q < p < r
\\ \iff& q < r
\\ \iff& q \in \mathbb{Q}
\end{align*}이다.

\textbf{(절 3)} 만약
\begin{align*}
p = \frac{c}{d} \in r^* \tag{$c \in \mathbb{Z}, d \in P_{\mathbb{Z}}$}
\end{align*}라고 하면, 먼저 $p < r$임을 이용하여,
\begin{align*}
p < r \iff& \frac{c}{d} < \frac{a}{b}
\\ \iff& 0 < \frac{a}{b} - \frac{c}{d}
\\ \iff& 0 < \frac{ad+cb}{bd}
\\ \iff& 0 < ad+cb \tag{$\because 0 < bd$}
\end{align*}인 것을 알 수 있다.

이제,
\begin{align*}
s = \frac{a+c}{b+d} \in \mathbb{Q}
\end{align*}를 생각하면,
\begin{align*}
p < s \iff& \frac{c}{d} < \frac{a+c}{b+d}
\\ \iff& 0 < \frac{a+c}{b+d} + (- \frac{c}{d})
\\ \iff& 0 < \frac{a+c}{b+d} + \frac{-c}{d}
\\ \iff& 0 < \frac{(a+c)d+(-c)(b+d)}{(b+d)d}
\\ \iff& 0 < \frac{(ad+cd-bc-cd}{(b+d)d}
\\ \iff& 0 < \frac{(ad-bc)}{(b+d)d}
\\ \iff& 0 < ad-bc \tag{$0 < (b+d)d$}
\end{align*}이므로, $p<s$는 성립한다. 또한,
\begin{align*}
s < r \iff& \frac{a+c}{b+d} < \frac{a}{b}
\\ \iff& 0 < \frac{a}{b} + (-\frac{a+c}{b+d})
\\ \iff& 0 < \frac{a}{b} + \frac{-a-c}{b+d}
\\ \iff& 0 < \frac{a(b+d)+(-a-c)(b)}{d(b+d)}
\\ \iff& 0 < \frac{ab+ad-ab-bc}{(b+d)d}
\\ \iff& 0 < \frac{ad-bc}{(b+d)d}
\\ \iff& 0 < ad-bc \tag{$0 < (b+d)d$}
\end{align*}이므로, $s<r$ 또한 성립한다.

따라서, $p<s$이고 $s \in r^*$인 $s$가 존재한다.

\textbf{결론.} 임의의 $r \in \mathbb{Q}$에 대해 $r^*$가 절단이다.

\section{2.3.3.}
\textbf{결합법칙.} 임의의 $\alpha, \beta, \gamma \in \mathbb{R}$에 대해
\begin{align*}
(\alpha + \beta) + \gamma = \alpha + (\beta + \gamma)
\end{align*}가 성립한다. 
\begin{align*}
x \in (\alpha + \beta) + \gamma \iff& \exists n \in (\alpha + \beta), c \in \gamma \mid x = n + c
\\ \iff& \exists a \in \alpha, b \in \beta, c \in \gamma \mid x = a + b + c
\\ \iff& \exists a \in \alpha, m \in (\beta + \gamma) \mid x = a + m
\\ \iff& x \in \alpha + (\beta + \gamma)
\end{align*}이므로, $(\alpha + \beta) + \gamma = \alpha + (\beta + \gamma)$이다.

\textbf{교환법칙.} 임의의 $\alpha, \beta \in \mathbb{R}$에 대해
\begin{align*}
\alpha + \beta = \beta + \alpha
\end{align*}가 성립한다. 
\begin{align*}
x \in \alpha + \beta \iff& \exists a \in \alpha, b \in \beta \mid x = a + b
\\ \iff&  \exists b \in \beta, \exists a \in \alpha \mid x = b + a
\\ \iff& x \in \beta + \alpha
\end{align*}이므로, $\alpha + \beta = \beta + \alpha$이다.

\section{2.3.6.}
\subsection{$\alpha 1^* = 1^* \alpha = \alpha$ 임을 보이자.}
\textbf{교환법칙. ($P_{\mathbb{R}}$)} 임의의 $\alpha, \beta \in P_{\mathbb{R}}$에 대해
\begin{align*}
\alpha \beta = \beta \alpha
\end{align*}가 성립한다. 
\begin{align*}
x \in \alpha \beta \iff& \exists a \in \alpha, b \in \beta \mid x \le ab
\\ \iff&  \exists b \in \beta, \exists a \in \alpha \mid x = ba
\\ \iff& x \in \beta \alpha
\end{align*}이므로, $\alpha \beta = \beta \alpha$이다.

\textbf{교환법칙. ($\mathbb{R}$)} 임의의 $\alpha, \beta \in \mathbb{R}$에 대해
\begin{align*}
\alpha \beta = 
\left\{\begin{matrix}
0* &=& 0* & \text{$\alpha = 0 \text{ or } \beta = 0$}
\\ \alpha \beta &=&\beta \alpha & \text{$\alpha \in P_{\mathbb{R}}, \beta \in P_{\mathbb{R}}$}
\\ -\alpha (-\beta) &=& -(-\beta) \alpha & \text{$\alpha \in P_{\mathbb{R}}, \beta \in -P_{\mathbb{R}}$}
\\ -(-\alpha) \beta &=& -\beta (-\alpha) & \text{$\alpha \in -P_{\mathbb{R}}, \beta \in P_{\mathbb{R}}$}
\\ (-\alpha) (-\beta) &=& (-\beta) (-\alpha) & \text{$\alpha \in -P_{\mathbb{R}}, \beta \in -P_{\mathbb{R}}$}
\end{matrix}\right.
\end{align*}이므로, $\alpha \beta = \beta \alpha$인 것을 알 수 있다.

교환법칙에 의해서 
\begin{align*}
\alpha 1^* = 1^* \alpha \tag{1}
\end{align*}임을 알 수 있다. 

이제 임의의 실수 $\alpha \in \mathbb{R}$에 대해 $\alpha 1^* = \alpha$임을 보이자.

\subsubsection{Case 1. $\alpha = 0^*$}
\begin{align*}
\alpha 1^* = 0^* 1^* = 0^* = \alpha
\end{align*}이므로 성립한다.

\subsubsection{Case 2. $\alpha \in P_{\mathbb{R}}$}
\textbf{1. $\alpha \supset \alpha 1^*$ 이다.}
\begin{align*}
x \in \alpha 1^* \implies& \exists r \in \alpha \cap P_{\mathbb{R}}, \exists s \in 1^* \cap P_{\mathbb{R}} \mid x \le rs
\\ \implies& \exists r \in \alpha \cap P_{\mathbb{R}}, \exists s \in 1^* \cap P_{\mathbb{R}} \mid  x \le rs < r
\\ \implies& x \in \alpha \cap P_{\mathbb{R}}
\\ \implies& x \in \alpha
\end{align*}

\textbf{2. $\alpha \subset \alpha 1^*$ 이다.}
$p \in \alpha \implies \exists q \in \alpha \mid q > p$이다. 그러면,
\begin{align*}
\exists r=q \in \alpha \cap P_{\mathbb{R}}, \exists s=\frac{p}{r} \in 1^* \cap P_{\mathbb{R}} \mid x \le rs = q \left(\frac{p}{q} \right)
\end{align*}이므로, $p \in \alpha 1^*$이다.

1.와 2.에 의해서 
\begin{align*}
\alpha = \alpha 1^*
\end{align*}이다.

\subsubsection{Case 3. $\alpha \in -P_{\mathbb{R}}$}
$a \in P_{\mathbb{R}}$인 경우에 $\alpha 1^* = \alpha$가 성립함을 이용하자.
\begin{align*}
\alpha 1^* = -(-\alpha) 1^* = -(-\alpha) = \alpha
\end{align*}이므로, $\alpha \in -P_{\mathbb{R}}$인 경우에도
\begin{align*}
\alpha 1^* = \alpha
\end{align*}가 성립한다.


\textbf{결론.} 
\\$x \in \mathbb{R}$이면,
\begin{align*}
\begin{matrix}
x = 0^* & \text{ or } & x \in P_{\mathbb{R}} & \text{ or } & x \in -P_{\mathbb{R}}
\end{matrix}
\end{align*}이고, 각각의 경우 (Case 1, Case 2, Case 3)에 대해서 $x 1^* = x$ 가 성립하므로, 
임의의 실수 $\alpha \in\mathbb{R}$에 대해서,
\begin{align*}
\alpha 1^* = \alpha \tag{2}
\end{align*}가 성립함을 알 수 있다.

(1), (2)에 의하여 임의의 실수 $\alpha \in \mathbb{R}$
\begin{align*}
\alpha 1^* = 1^* \alpha = \alpha
\end{align*}이다. \qed

\subsection{$0^* < 1^*$임을 보이자.}
$\frac{1}{2} \notin 0^*$이고, $\frac{1}{2} \in 1^*$이므로,
\begin{align*}
0^* \neq 1^* \tag{1}
\end{align*}이다.
\begin{align*}
x \in 0^* \iff& x \in \mathbb{Q}, x < 0
\\ \implies& x \in \mathbb{Q}, x < 1
\\ \implies& x \in 1^*
\end{align*}이므로, 
\begin{align*}
0^* \subset 1^* \tag{2}
\end{align*}이다.

(1), (2)에 의해 $0^* \subsetneq 1^*$이고, 이는 
\begin{align*}
0^* < 1^*
\end{align*}을 의미한다. \qed


\section{2.3.7.}
\begin{align*}
\gamma = 0^* \cup \left\{0\right\} \cup \left\{q \in P_{\mathbb{Q}} \mid \left(\exists r \in P_{\mathbb{Q}} \mid r > q, \frac{1}{r} \notin \alpha \right) \right\}
\end{align*}
\subsection{$\gamma$는 절단이다.}
\subsubsection{(절1)}
\begin{align*}
0 \in \gamma \implies \gamma \neq \varnothing \tag{(1)}
\end{align*}이다. 또한,
\begin{align*}
\alpha \in P_{\mathbb{R}} \implies& 0 \in \alpha 
\\ \implies& \exists x \in \alpha \mid x \in P_{\mathbb{Q}}
\end{align*}이다. 

$\frac{1}{x} \in \alpha \cap P_{\mathbb{Q}}$가 $\gamma$에 속하지 않는 것을 증명해보자.

\textbf{pf. [귀류법]} $\frac{1}{x} \in \gamma$라 하자. 그러면,
\begin{align*}
x \in \gamma \iff& \exists r \in P_{\mathbb{Q}} \mid r > \frac{1}{x}, \frac{1}{r} \notin \alpha
\end{align*}이다. 하지만,
\begin{align*}
r > \frac{1}{x} \implies \frac{1}{r} < x \implies \frac{1}{r} \in \alpha
\end{align*}이므로, $\frac{1}{r} \in \alpha \text{ and } \frac{1}{r} \notin \alpha$이다. 따라서 모순이다.

따라서, $\frac{1}{x} \notin \gamma$이다. \qed

위를 이용하면
\begin{align*}
\frac{1}{x} \in P_{\mathbb{Q}}, \frac{1}{x} \notin \gamma \implies \gamma \neq \mathbb{Q} \tag{2}
\end{align*}을 알 수 있다.

(1), (2)에 의해 (절 1)은 성립한다.

\subsubsection{(절2)}
임의의 $p \in \gamma$를 생각해보자.

\textbf{Case 1. $p \le 0$인 경우}
$\gamma$의 정의에 의해 $p-1 < 0$이므로,
\begin{align*}
\exists q = p-1 < p \mid q \in \gamma
\end{align*}이다.

\textbf{Case 2. $p > 0$인 경우}
$$p \in \gamma \setminus \left\{0\right\} \setminus 0^*$$
이므로 성립함을 알 수 있다. 간략하게 적어보면,
\begin{align*}
p \in \gamma, p > q \implies & \left(\exists r \in P_{\mathbb{Q}} \mid r > p, \frac{1}{r} \notin \alpha \right), p > q
\\ \implies & \exists r \in P_{\mathbb{Q}} \mid r > q, \frac{1}{r} \notin \alpha
\\ \iff& q \in \gamma
\end{align*}이므로 (절 2)는 성립한다.

\subsubsection{(절3)}
$\alpha \in P_\mathbb{R}$이므로, $\alpha > 0^* \iff \alpha \supset 0^*, \alpha \neq 0^*$이다. 또한 $\alpha \neq \mathbb{Q}$이므로,
\begin{align*}
\mathbb{Q} \setminus \alpha =& \mathbb{Q} \setminus \alpha \setminus 0^* \setminus
\\ =& \mathbb{Q} \setminus \alpha \setminus 0^* \setminus \left\{0\right\} \tag{$0 \in \alpha$}
\end{align*}이므로, $\exists x \in \mathbb{Q} \mid x \in (\mathbb{Q} \setminus \alpha)$가 존재한다. 이 $x$는 
$$x > 0 \text{ and } x \notin \alpha$$이다. 이제 
$$u = \frac{1}{x+1} > 0$$가 $\gamma$에 속함을 보이자.

\textbf{pf.}
\begin{align*}
\frac{1}{x} > \frac{1}{x+1} = u \text{ and } x \notin \alpha \iff u \in \gamma \tag{1}
\end{align*}
 \qed

이제 임의의 $p \in \gamma$에 대해 (절 3)을 만족하는지 확인해보자.

\textbf{Case 1. $p \le 0$}
\begin{align*}
\exists q = u > p \mid u \in \gamma \tag{$\because$ (1)}
\end{align*}
 
\textbf{Case 2. $p > 0$}
\begin{align*}
p \in \gamma \implies \exists q \in P_{\mathbb{Q}} \mid q > p, \frac{1}{q} \notin \alpha
\end{align*}이다. 이제
\begin{align*}
x = \frac{(p + q)}{2} 
\end{align*}인 $ x \in P_{\mathbb{Q}}$ 생각하자. $p < x < q$는 만족한다는 것을 알 수 있다.
\begin{align*}
\frac{1}{q} \notin \alpha  \text{ and } x < q \implies& x \in \gamma
\end{align*}이다. 

따라서 $p \in \gamma \implies \left( \exists  x \in \gamma \mid x > p \right)$이므로 (절 3)을 만족한다.

(절1), (절2), (절3)을 모두 만족하므로 $\gamma$는 절단이다. \qed



\subsection{$\alpha \gamma = 1^*$이다.}
\subsubsection{1. $\alpha \gamma \subset 1^*$이다.}
\begin{align*}
x \in \alpha \gamma \iff& \exists r \in \alpha \cap P_{\mathbb{Q}}, \exists s \in \gamma \cap P_{\mathbb{Q}} \mid x \le rs
\end{align*}이다. 여기서 $s \in \gamma \cap P_{\mathbb{Q}}$이므로,
\begin{align*}
\exists k \in P_{\mathbb{Q}} \mid k > s, \frac{1}{k} \notin \alpha
\end{align*}이고, 
\begin{align*}
\left( \frac{1}{k} \notin \alpha, r \in \alpha \right) \text{ and } s < k \implies r < \frac{1}{k} < \frac{1}{s}
\end{align*}를 얻는다. 따라서,
\begin{align*}
x \in \alpha \gamma \iff& \exists r \in \alpha \cap P_{\mathbb{Q}}, \exists s \in \gamma \cap P_{\mathbb{Q}} \mid x \le rs
\\ \implies & \exists r \in \alpha \cap P_{\mathbb{Q}}, \exists s \in \gamma \cap P_{\mathbb{Q}} \mid x < \left(\frac{1}{s} \right) s
\\ \implies & \exists r \in \alpha \cap P_{\mathbb{Q}}, \exists s \in \gamma \cap P_{\mathbb{Q}} \mid x < 1
\\ x \in 1^*
\end{align*}이므로, $\alpha \gamma \subset 1^*$이다.

\subsubsection{2. $\alpha \gamma \supset 1^*$이다.}

$\alpha \in P_\mathbb{R}$이므로, $\alpha > 0^* \iff \alpha \supset 0^*, \alpha \neq 0^*$이다. 따라서, $$\exists x \in (\alpha \setminus  0^*) \implies x \ge 0$$
이고, 만약 $x = 0$라면 절단의 성질에의해 $\exists y \in \alpha \mid y > x = 0$가 존재한다. 따라서 $\alpha$에서 양의 유리수 
\begin{align*}
\exists a > 0 \in \alpha \tag{1}
\end{align*}가 존재한다고 할 수 있다.


양의 실수($P_{\mathbb{R}}$)의 곱하기의 정의에 의해서 $\alpha \gamma \supset 0^* \cup \left\{0\right\}$이다. 

그러므로, $\alpha \gamma \supset 1^* \setminus \left( 0^* \cup \left\{0\right\} \right)$을 보이면 충분하다. 여기서 
\begin{align*}
1^* \setminus \left( 0^* \cup \left\{0\right\} \right) =& \left\{p \in \mathbb{Q} \mid p < 1 \right\} \setminus \left( \left\{p \in \mathbb{Q} \mid p < 0 \right\} \cup \left\{0\right\} \right)
\\ =& \left\{p \in \mathbb{Q} \mid 0 < p < 1 \right\}
\end{align*}이므로, $\alpha \gamma \supset \left\{p \in \mathbb{Q} \mid 0 < p < 1 \right\}$을 보이는 것과 동치이다.

\textbf{$\alpha \gamma \supset \left\{p \in \mathbb{Q} \mid 0 < p < 1 \right\}$을 보이자.}
\\ \textbf{pf.} $\forall s \in \left\{p \in \mathbb{Q} \mid 0 < p < 1 \right\}$에 대해 생각하자. 

이제 임의의 원소 $0 < s < 1$에 대해
$$ t = \frac{1-s}{2} a $$라고 하자. 그리고 집합 $A$를

\begin{align*}
A = \left\{n \in \mathbb{N} \mid nt \in \alpha  \right\}
\end{align*}라고 정의하자. $A$가 최대 원소를 가진다는 것을 보이기 위해 다음 두가지를 증명하겠다.

\textbf{1. $A$는 공집합이 아니다.}
$a \in \alpha, a > 0 \implies 0 \in \alpha$이다. 그러면 
$$ 0 = 0t \in \alpha \implies 0 \in A$$이다.
따라서, $A$는 공집합이 아니다.

\textbf{2. $A$는 위로 유계이다.}
\\ \textbf{pf.[귀류법]} $A$가 위로 유계가 아니라고 가정하자. 

유리수체는 아르키메데스 성질을 만족하므로, 임의의 유리수 $q \in \mathbb{Q}$에 대해,
\begin{align*}
\exists k \in \mathbb{N} \mid  q < kt
\end{align*}인 $k$가 존재한다. 여기서 $A$는 상계를 가지지 않으므로 $k$는 상계가 아니므로
\begin{align*}
\exists n \in A \mid n > k 
\end{align*}이고, 이를 이용하면
\begin{align*}
q < kt < nt \in \alpha
\end{align*}이다. 따라서 $q \in \alpha$가 된다. 이 결과로 $\alpha = \mathbb{Q}$가 되며 이는 절단의 정의에 모순이다. 

따라서 $A$는 위로 유계이다. \qed

위 두 증명에 의해서 집합 $A$는 최대 원소를 가진다. 이 최대원소를 $n_0 \in \mathbb{N}$이라고 하자. 그러면
\begin{align*}
&{n_0}t \in \alpha \cap P_{\mathbb{Q}}
\\ &{n_0 + 1}t \notin \alpha \cap P_{\mathbb{Q}}
\\ &{n_0 + 2}t \notin \alpha \cap P_{\mathbb{Q}}
\end{align*}이므로, 이를 이용하면
\begin{align*}
\frac{1}{(n_0 + 1)t} > \frac{1}{(n_0 + 2)t} \text{ and } 1 / \frac{1}{(n_0 + 1)t} = (n_0 + 1)t \notin \alpha
\end{align*}이므로 
\begin{align*}
\frac{1}{(n_0 + 2)t} \in \gamma
\end{align*}이다. 이제
\begin{align*}
&x = (n_0)t \in \alpha, y = \frac{1}{(n_0 + 2)t}  \in \gamma \text{이므로,}
\\&s \le  xy = \frac{(n_0)t}{(n_0+2)t} = \frac{n_0}{n_0+2} \implies s \in \alpha \gamma
\end{align*} 이므로, $$s \le \frac{n_0}{n_0+2}$$을 보이면 증명이 끝난다. 이를 보이자.
\begin{align*}
(n_0+2)t \notin \alpha, a \in \alpha \implies& a < (n_0+2)t
\\ \implies& a < (n_0+2) \frac{1-s}{2} a
\\ \implies& 1 < (n_0+2) \frac{1-s}{2} \tag{$a > 0$}
\\ \implies& \frac{2}{n_0+2} < 1-s 
\\ \implies& s < 1 -  \frac{2}{n_0+2}
\\ \implies& s \le \frac{n_0}{n_0+2}
\end{align*}이므로 $$s \le \frac{n_0}{n_0+2}$$는 성립한다.

따라서, $s \in \alpha \gamma$이고 이는
$$\alpha \gamma \supset 1^*$$을 의미한다. \qed





\textbf{결론.} $\alpha \gamma \subset 1^*$이며 $\alpha \gamma \supset 1^*$이므로,
\begin{align*}
\alpha \gamma = 1^*
\end{align*}이다.

\end{document}





\iffalse
\textbf{Case 1. $q \in \gamma$ 인 경우}
\\ $p \in \gamma \implies \left( \exists  q \in \gamma \mid q > p \right)$이므로 만족한다.

\textbf{Case 2. $q \notin \gamma$인 경우}
\begin{align*}
q \notin \gamma \iff& \nexists r \in P_{\mathbb{Q}} \mid r > q \text{ and } \frac{1}{r} \notin \alpha
\\ \iff& \forall r \in P_{\mathbb{Q}} \mid r \le q \text{ or } \frac{1}{r} \in \alpha
\\ \iff& \forall q < r \in P_{\mathbb{Q}} \mid \frac{1}{r} \in \alpha
\\ \iff&  \forall q < r \in P_{\mathbb{Q}} \mid \left( \frac{1}{r} < \frac{1}{q} \text{ and } 0 < r \mid \frac{1}{r} \in \alpha \right)
\\ \iff& \alpha \supset \left\{ \frac{1}{r} \in P_{\mathbb{Q}} \mid \frac{1}{r} < \frac{1}{q} \right\}  \tag{1}
\end{align*}이다. 또한, 가정에 의해 $\alpha \in P_{\mathbb{R}}$이므로,
\begin{align*}
\alpha \supset \left( 0^* \cup \left\{ 0 \right\} \right) \tag{2}
\end{align*}이다. 또한, $\frac{1}{q} \notin \alpha$라는 가정을 이용하면,
\begin{align*}
\frac{1}{q} \notin \alpha, q \le \forall r \in P_{\mathbb{Q}} \implies& \frac{1}{q} \notin \alpha, \frac{1}{q} \le \frac{1}{r} \tag{$\because q \in P_{\mathbb{Q}}$}
\\ \implies& \frac{1}{r} \notin \alpha
\end{align*}를 알 수 있다. 이를 이용하면,
\begin{align*}
\alpha \cap \left\{ \frac{1}{r} \in P_{\mathbb{Q}} \mid \frac{1}{q} \le \frac{1}{r} \right\} = \varnothing \tag{3}
\end{align*}를 얻는다.

(1), (2), (3)에 의해 
\begin{align*}
\alpha = \left( \frac{1}{q} \right) ^ *
\end{align*}임을 알 수 있다. 

이제 




\textbf{Claim. $\exists n \in \mathbb{N} \mid a^n > b \text{ }(a \in \left\{p \in \mathbb{Q} \mid p > 1 \right\}, b \in P_{\mathbb{Q}})$}



먼저 유리수체는 아르키메데스 성질을 가짐을 상기하자.
\begin{align*}
a-1 > 0 \implies&  1 \le \exists x \in \mathbb{N} \mid (a-1) > \frac{1}{x}
\\ \implies&  1 \le \exists x \in \mathbb{N} \mid a > 1 + \frac{1}{x} \tag{1}
\end{align*}이므로, $a > 1 + \frac{1}{x}$을 만족하는 자연수 $x$가 존재한다. 또한,
\begin{align*}
\exists y \in \mathbb{N} \mid 1 + \frac{y}{x} > b
\end{align*}을 만족하는 자연수 $y$가 존재한다. 그 이유는 $b \le 1$인 경우에는 $\exists y = 1$인 경우에 성립하고, $b > 1$인 경우에는 \begin{align*}
b-1 > 0 \implies& 1 \le \exists y \in \mathbb{N} \mid y > x(b-1)
\\ \implies& 1 \le \exists y \in \mathbb{N} \mid x + y > xb
\\ \implies& 1 \le \exists y \in \mathbb{N} \mid 1 + \frac{y}{x} > b \tag{2}
\end{align*}이기 때문이다.

마지막으로,
\begin{align*}
\left(1 + \frac{1}{x} \right)^y \ge 1 + \frac{y}{x} \tag{$x \in \mathbb{N} \setminus \left\{0\right\}, y \in \mathbb{N}$}
\end{align*}를 증명하자. $y$에 대한 수학적 귀납법으로 보일 것이다.
\\\textbf{pf.} $X = \left\{ n \in \mathbb{N} \mid \left(1 + \frac{1}{x} \right)^n \ge 1 + \frac{n}{x} \right\}$라 하자.
\begin{align*}
\left(1 + \frac{1}{x} \right)^0 = 1 \ge 1 = 1 + \frac{0}{x}
\end{align*}이므로 $0 \in X$이다. 이제 $n \in X$라고 가정하면,
\begin{align*}
\left(1 + \frac{1}{x} \right)^{n+1} =& \left(1 + \frac{1}{x} \right)^n \left(1 + \frac{1}{x} \right)
\\ \ge& \left(1 + \frac{n}{x}\right) \left(1 + \frac{1}{x} \right)
\\ =& \left(1 + \frac{1}{x} \right) + \left(\frac{n}{x} + \frac{n}{x^2} \right)
\\ \ge& \left(1 + \frac{n+1}{x} \right)
\end{align*}이므로 $n^+ \in X$이다. 따라서 수학적 귀납법에 의해 $X = \mathbb{N}$이고, 임의의 자연수 $n \in \mathbb{N}$에 대해
\begin{align*}
\left(1 + \frac{1}{x} \right)^y \ge 1 + \frac{y}{x} \tag{3}
\end{align*}가 성립한다. \qed

이제 (1), (2), (3)을 이용하면, 자연수 $y \in \mathbb{N}$이 존재하여
\begin{align*}
a^y > \left(1 + \frac{1}{x} \right)^y \ge 1 + \frac{y}{x} > b
\end{align*}를 만족한다. \qed

이제
\begin{align*}
\alpha \gamma \supset 1^*
\end{align*}을 증명해보자.

양의 실수($P_{\mathbb{R}}$)의 곱하기의 정의에 의해서 $\alpha \gamma \supset 0^* \cup \left\{0\right\}$이다. 

그러므로, $\alpha \gamma \supset 1^* \setminus \left( 0^* \cup \left\{0\right\} \right)$을 보이면 충분하다. 여기서 
\begin{align*}
1^* \setminus \left( 0^* \cup \left\{0\right\} \right) =& \left\{p \in \mathbb{Q} \mid p < 1 \right\} \setminus \left( \left\{p \in \mathbb{Q} \mid p < 0 \right\} \cup \left\{0\right\} \right)
\\ =& \left\{p \in \mathbb{Q} \mid 0 < p < 1 \right\}
\end{align*}이므로, $\alpha \gamma \supset \left\{p \in \mathbb{Q} \mid 0 < p < 1 \right\}$을 보이는 것과 동치이다.

\textbf{$\alpha \gamma \supset \left\{p \in \mathbb{Q} \mid 0 < p < 1 \right\}$을 보이자.}
\\ \textbf{pf.} $\forall s \in \left\{p \in \mathbb{Q} \mid 0 < p < 1 \right\}$에 대해 생각하자. $t = s^{-2}$라 하고,
\begin{align*}
A = \left\{n \in \mathbb{Z} \mid t^n \in \alpha  \right\}
\end{align*}라고 정의하자. $A$가 최대 원소를 가진다는 것을 보이기 위해 다음 두가지를 증명하겠다.

\textbf{1. $A$는 공집합이 아니다.}
\\ $\alpha \supset 0^*$이므로 $a \in \alpha \setminus 0^*$를 하나 고르자, 그러면
\begin{align*}
&\exists m \in \mathbb{N} \mid t^{m} > \frac{1}{a}  \tag{By Claim}
\\ \iff &\exists m \in \mathbb{N} \mid t^{-m} < a
\end{align*}이다. 

따라서 $t^{-m} \in \alpha$이고, $-m \in A$이므로 $A$는 공집합이 아니다.

\textbf{2. $A$는 위로 유계이다.}
\\ \textbf{pf.[귀류법]} $A$가 위로 유계가 아니라고 가정하자. 

그러면 임의의 유리수 $q \in \mathbb{Q}$에 대해
\begin{align*}
\exists k \in \mathbb{N} \mid  q < t^k \tag{By Claim / $\text{when }q \le 0\text{, choose } k = 0$ }
\end{align*}인 $k$가 존재한다. 여기서 $A$는 상계를 가지지 않으므로 $k$는 상계가 아니므로
\begin{align*}
\exists k < n \in A 
\end{align*}이고, 이를 이용하면
\begin{align*}
q < t^k < t^n \in \alpha
\end{align*}이다. 따라서 $q \in \alpha$가 된다. 이 결과로 $\alpha = \mathbb{Q}$가 되며 이는 절단의 정의에 모순이다. 

따라서 $A$는 위로 유계이다. \qed

위 두 증명에 의해서 집합 $A$는 최대 원소를 가진다. 이 최대원소를 $n_0 \in \mathbb{N}$이라고 하자. 그러면
\begin{align*}
&t ^ {n_0} \in \alpha \cap P_{\mathbb{Q}}
\\ &t ^ {n_0 + 1} \notin \alpha \cap P_{\mathbb{Q}}
\\ &t ^ {n_0 + 2} \notin \alpha \cap P_{\mathbb{Q}}
\end{align*}이므로, 이를 이용하면
\begin{align*}
\frac{1}{t ^ {n_0 + 1}} > \frac{1}{t ^ {n_0 + 2}} \text{ and } 1 / \frac{1}{t ^ {n_0 + 1}} = t ^ {n_0 + 1} \notin \alpha
\end{align*}이므로 
\begin{align*}
\frac{1}{t ^ {n_0 + 2}} \in \gamma
\end{align*}이다. 따라서,
\begin{align*}
&x = t ^ {n_0} \in \alpha, y = \frac{1}{t ^ {n_0 + 2}}  \in \gamma
\\&s = t^{-2} = t ^ {n_0} \frac{1}{t ^ {n_0 + 2}} = x y
\end{align*}이므로 $s \in \alpha \gamma$이다. 

따라서, $\alpha \gamma \supset 1^*$이다. \qed
\fi



































