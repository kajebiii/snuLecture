%%%%%%%%%%%%%%%%%%%%%%%%%%%%%%%%%%%%%%%%%
% Programming/Coding Assignment
% LaTeX Template
%
% This template has been downloaded from:
% http://www.latextemplates.com
%
% Original author:
% Ted Pavlic (http://www.tedpavlic.com)
%
% Note:
% The \lipsum[#] commands throughout this template generate dummy text
% to fill the template out. These commands should all be removed when 
% writing assignment content.
%
% This template uses a Perl script as an example snippet of code, most other
% languages are also usable. Configure them in the "CODE INCLUSION 
% CONFIGURATION" section.r
%
%%%%%%%%%%%%%%%%%%%%%%%%%%%%%%%%%%%%%%%%%

%----------------------------------------------------------------------------------------
%	PACKAGES AND OTHER DOCUMENT CONFIGURATIONS
%----------------------------------------------------------------------------------------

\documentclass{article}
\usepackage[hangul]{kotex}
\usepackage{fancyhdr} % Required for custom headers
\usepackage{lastpage} % Required to determine the last page for the footer
\usepackage{extramarks} % Required for headers and footers
\usepackage[usenames,dvipsnames]{color} % Required for custom colors
\usepackage{graphicx} % Required to insert images
\usepackage{listings} % Required for insertion of code
\usepackage{courier} % Required for the courier font
\usepackage{lipsum} % Used for inserting dummy 'Lorem ipsum' text into the template
\usepackage{amsthm,amsmath}
\usepackage[table,xcdraw]{xcolor}

\usepackage{verbatim} % Required for multiple comment
\usepackage{amsmath} % Required for use \therefore \because and others..
\usepackage{amssymb} % Required for use \therefore \because and others..
\usepackage{algorithm, algpseudocode}
\usepackage{verbatim} % for commment, verbatim environment
\usepackage{spverbatim} % automatic linebreak verbatim environment
\usepackage{listings}
\usepackage{ulem}
\usepackage{hyperref}
\DeclareGraphicsExtensions{.pdf,.png,.jpg}

\usepackage{xcolor}




% Margins
\linespread{1.25} % Line spacing
\usepackage[a4paper,top=2cm,bottom=1cm,left=1cm,right=1cm,marginparwidth=1.75cm]{geometry}
\usepackage{titlesec,sectsty}
\setlength\parindent{0pt}
\setlength\parskip{10pt}
\setlength{\abovedisplayskip}{3pt}
\setlength{\belowdisplayskip}{3pt}
\sectionfont{\fontsize{12}{10}\selectfont}
\subsectionfont{\fontsize{10}{10}\selectfont}
\titlespacing{\subsection}{0pt}{-\parskip}{-\parskip}
\usepackage{multicol}
\usepackage{cuted}
\newenvironment{Figure}
{\par\medskip\noindent\minipage{\linewidth}}
{\endminipage\par\medskip}


% mathematics
\usepackage{amsmath,amssymb,mathtools}
\usepackage{cancel} % cancelation of terms
\usepackage{nicefrac} % certain fraction styles
\newcommand*\diff{\mathop{}\!\mathrm{d}} % differential
\DeclareMathOperator{\arsinh}{arsinh}

% Set up the header and footer
\pagestyle{fancy}
\lhead{JongBeom Kim}
\chead{현대대수학 HW 1} % Top center head
\rhead{2018 Spring} % Top right header
\lfoot{\lastxmark} % Bottom left footer
\cfoot{} % Bottom center footer
\renewcommand\headrulewidth{0.4pt} % Size of the header rule
\renewcommand\footrulewidth{0.4pt} % Size of the footer rule
\newcommand{\myul}[2][black]{\setulcolor{#1}\ul{#2}\setulcolor{black}}

\setlength\parindent{0pt} % Removes all indentation from paragraphs

%----------------------------------------------------------------------------------------
%	CODE INCLUSION CONFIGURATION
%----------------------------------------------------------------------------------------


\setcounter{secnumdepth}{0} % Removes default section numbers
\newcounter{homeworkProblemCounter} % Creates a counter to keep track of the number of problems


\renewcommand\headrule
{
	\begin{minipage}{1\textwidth}
		\hrule width \hsize height 1pt \kern 1.5pt \hrule width \hsize height 0.5pt  
	\end{minipage}\par
}%

% lists
\usepackage{enumitem}
\setlist[itemize]{topsep=-10pt} %set spacing for itemize
\setitemize{itemsep=3pt}
\setlist[enumerate]{topsep=-10pt} %set spacing for itemize
\setenumerate{itemsep=3pt}

%----------------------------------------------------------------------------------------
%	TITLE PAGE
%----------------------------------------------------------------------------------------

%----------------------------------------------------------------------------------------


\begin{document}
\twocolumn



\section{Homework 1}

\subsection{3.26.}
함수 $\phi : S \rightarrow S'$은 1-1, onto 이므로, 역함수 $\phi^{-1} : S' \rightarrow S' (\text{1-1, onto})$가 존재한다. 그러므로, 임의의 $x', y' \in S'$에 대하여
\begin{align*}
\exists x, y \in S | x = \phi^{-1}(x'), y = \phi^{-1}(y')
\end{align*}
이를 이용하면, 
\begin{align*}
\phi^{-1}(x') * \phi^{-1}(y') =& x * y
\\ =& \phi^{-1}(\phi(x * y))
\\ =& \phi^{-1}(\phi(x) *' \phi(y)) &\text{($\phi$는 동형사상)}
\\ =& \phi^{-1}(x' *' y')
\end{align*}
따라서, $\phi^{-1}$은 $<S', *'>$에서 $<S, *>$위로의 동형사상이다.

\subsection{3.27.}
$\psi \circ \phi$은 전단사함수의 합성함수이므로 전단사함수이다. 모든 $x, y \in S$에 대해서,
\begin{align*}
\psi \circ \phi(x * y) =& \psi(\phi(x) *' \phi(y)) &\text{($\phi$가 동형사상)}
\\ =& \psi \circ \phi(x) *^{''} \psi \circ \phi(y) &\text{($\psi$가 동형사상)}
\end{align*}
따라서, $\psi \circ \phi$은 $<S, *>$에서 $<S^{''}, *^{''}>$ 위로의 동형사상이다.

\subsection{3.33.}
\subsubsection{a.}
\begin{align*}
\phi(a+bi) = \begin{bmatrix}a & -b \\  b & a \end{bmatrix}
\end{align*}
$\phi$는 전단사함수이다. 임의의 $x = a+bi, y = c+di \in \mathbb{C}$에 대해
\begin{align*}
\phi(x + y) =& \phi((a+c) + (b+d)i)
\\ =& \begin{bmatrix}a+c & -b-d \\  b+d & a+c \end{bmatrix}
\\ =& \begin{bmatrix}a & -b \\  b & a \end{bmatrix} + \begin{bmatrix}c & -d \\  d & c \end{bmatrix}
\\ =& \phi(x) + \phi(y)
\end{align*}
따라서, $<\mathbb{C}, +>$와 $<\mathbb{H}, +>$는 동형이다.
\subsubsection{b.}
\begin{align*}
\phi(a+bi) = \begin{bmatrix}a & -b \\  b & a \end{bmatrix}
\end{align*}
$\phi$는 전단사함수이다. 임의의 $x = a+bi, y = c+di \in \mathbb{C}$에 대해
\begin{align*}
\phi(x \cdot y) =& \phi((ac-bd) \cdot (bc+ad)i)
\\ =& \begin{bmatrix}(ac-bd) & -(bc+ad) \\  (bc+ad) & (ac-bd) \end{bmatrix}
\\ =& \begin{bmatrix}a & -b \\  b & a \end{bmatrix} \cdot \begin{bmatrix}c & -d \\  d & c \end{bmatrix}
\\ =& \phi(x) \cdot \phi(y)
\end{align*}
따라서, $<\mathbb{C}, \cdot>$와 $<\mathbb{H}, \cdot>$는 동형이다.

\subsection{4.29.}
$\forall a \in G, a * a \neq e$라면, $\exists b \in G, a * b = b * a = e$이다.

또한, $b \neq c$인 $b, c \in G$에 대해 $a * b = a * c = e$일 수 없다.

짝수개의 원소를 갖는 군 $G$에 대해서, 만일 $a * a = e$를 만족하는 $G$의 원소 $a \neq e$가 존재하지 않는다면, 위 두 성질에 의해서 $e$를 제외한 홀수개의 원수가 짝을 이뤄야한다. 이는 불가능하므로 $a * a = e$를 만족하는 $G$의 원소 $a$가 존재한다.

\subsection{4.30.}
\subsubsection{a.}
$a, b, c \in R^*$에 대해 $(a * b) * c = a * (b * c)$임을 보이자.
\begin{align*}
(a * b) * c =& (|a|b) * c 
\\ =& \left| |a| b \right| c
\\ =& |a||b| c
\\ =& |a| (b * c)
\\ =& a * (b * c)
\end{align*}

\subsubsection{b.}
$e = 1 \text{ or } -1$이면 $e * a = a$이다.

$a > 0$이면 $a * 1 = a$이고, $a < 0$라면 $a * -1 = a$이다.

\subsubsection{c.}

결합법칙은 성립하지만, 모든 원소에 대한 공통된 항등원 $e$가 존재하지 않는다.

따라서 $R^*$는 이 이항연산을 갖는 군이 될 수 없다.

\subsubsection{d.}
각 원소에 대한 항등원이 존재하더라도, 모든 원소의 공통된 항등원이 아니라면 군이 될 가능성은 없다.

\subsection{4.31.}
\begin{align*}
x * x =& x
\\ x^{-1} * x * x =& x^{-1} * x
\\ e * x =& e
\\ x =& e
\end{align*}
따라서, $x=e$이고 군에서 항등원은 유일하다.

\subsection{4.32.}
\begin{align*}
(a * b) * (a * b) =& e
\\ a * (b * a) * b =& e
\\ b * a =& a^{-1} *  b^{-1}
\\ b * a =& a * b
\end{align*}
따라서, $\forall x \in G | x * x = e$인 군 $G$은 가환이다.
\subsection{4.37.}
\begin{align*}
(b * c * a) * (b * c * a) =& b * c * (a * b * c) * a
\\ =& b * c * a
\end{align*}
이다. 따라서, 4.31.에 의해
\begin{align*}
b * c * a = e
\end{align*}
이다.
\subsection{4.39.}

\subsubsection{1. *은 결합법칙이 성립한다.}

\subsubsection{2. $\forall x \in G | \exists e \in G, e * x = x$}
\textbf{pf.}
어떤 $a \in G$에 대해, $\exists e \in G, e * a = a$이다.
\begin{align*}
&\forall x \in G, \exists y \in G | a * y = x \text{이므로}
\\& e * a * y = a * y
\\& \forall x \in G, e * x = x
\end{align*}

\subsubsection{3. $\forall a \in G | \exists a^{-1} \in G, a^{-1} * a = e$}

\subsubsection{좌공리 1, 2, 3이 성립하면 군이다.}
\textbf{pf.}
\begin{align*}
(a * a^{-1}) * (a * a^{-1}) &= a * (a^{-1} * a) * a^{-1}
\\ &= (a * a^{-1})
\end{align*}
이므로, 4.31.에 의해 $a * a^{-1} = e$이다.

또,
\begin{align*}
a * e =& a * (a^{-1} * a)
\\ =& (a * a^{-1}) * a
\\ =& e * a
\\ =& a
\end{align*}
이므로, 좌공리 1, 2, 3이 성립하면 군이다. \qed

따라서, 집합 $G$은 군이다.



\subsection{5.34.}
\begin{align*}
A =& \begin{bmatrix}0 & 0 & 0 & 1\\ 0 & 0 & 1 & 0\\ 1 & 0 & 0 & 0\\ 0 & 1 & 0 & 0\end{bmatrix}
\\ A^2 =& \begin{bmatrix}0 & 1 & 0 & 0\\ 1 & 0 & 0 & 0\\ 0 & 0 & 0 & 1\\ 0 & 0 & 1 & 0\end{bmatrix}
\\ A^3 =& \begin{bmatrix}0 & 0 & 1 & 0\\ 0 & 0 & 0 & 1\\ 0 & 1 & 0 & 0\\ 1 & 0 & 0 & 0\end{bmatrix}
\\ A^4 =& \begin{bmatrix}1 & 0 & 0 & 0\\ 0 & 1 & 0 & 0\\ 0 & 0 & 1 & 0\\ 0 & 0 & 0 & 1\end{bmatrix} = I
\end{align*}
위수는 4이다.


\subsection{5.42.}
동형사상이므로, $\phi$는 전단사함수이다.

\textbf{성질 1. $e_{G'} = \phi(e_G)$}
\begin{align*}
\phi(x) =& \phi(e_G * x)
\\ =& \phi(e_G) * \phi(x)
\end{align*}
$\therefore e_{G'} = \phi(e_G)$

\textbf{성질 2. $\phi(a^{-1}) = \phi(a)^{-1}$}
\begin{align*}
e_{G'} =& \phi(a * a^{-1}) 
\\ =& \phi(a) * \phi(a^{-1})
\end{align*}
$\therefore \phi(a^{-1}) = \phi(a)^{-1}$

\textbf{성질 3. $\forall n \in \mathbb{Z} | \phi(a^n) = \phi(a)^n$}

1. $n=0$인 경우

$\phi(a^0) = \phi(e_G) = e_{G'} = \phi(a) ^ 0$
이므로 성립한다.

2. $n=k$가 성립한다고 가정하고 $n=k+1$인 경우에 성립하는지 보자.
\begin{align*}
\phi(a^{k+1}) =& \phi(a * a^k)
\\ =& \phi(a) *' \phi(a^k)
\\ =& \phi(a) *' \phi(a)^k
\\ =& \phi(a)^{k+1}
\end{align*}
이므로, 성립한다.

3. $n=k$가 성립한다고 가정하고 $n=k-1$인 경우에 성립하는지 보자.
\begin{align*}
\phi(a^{k-1}) =& \phi(a^{-1} * a^k)
\\ =& \phi(a^{-1}) *' \phi(a^k)
\\ =& \phi(a)^{-1} *' \phi(a)^k
\\ =& \phi(a)^{k-1}
\end{align*}
이므로, 성립한다.

$\therefore \forall n \in \mathbb{Z} |  \phi(a^n) = \phi(a)^n$이다.

\textbf{결론} 동형사상 $\phi$가 존재하면, $G = \left\{a^n | n \in \mathbb{Z} \right\}$일 때, $G' = \left\{\phi(a)^n | n \in \mathbb{Z}  \right\}$이다.

따라서, $G$가 순환적이면 $G'$도 순환적이다.


\subsection{5.49.}
\textbf{귀류법} 만약, $a^n = e$가 되는 $n \in \mathbb{Z}^{+}$가 존재하지 않는다고 가정하자.
그러면 $x, y \in \mathbb{Z}, x > y$인 $x, y$에 대해 $a^x \neq a^y$이다. 왜냐하면 $a^{x-y} \neq e$이기 때문이다.

$\therefore |\langle a \rangle| = \infty$

하지만,  $\langle a \rangle \leq G$이고, $|G| < \infty$이므로 모순이다. 

따라서, $\forall a \in G, \exists n \in \mathbb{Z}^{+} | a^n = e$

\subsection{5.57. }
\textbf{대우명제} 순환군이 아닌 군은 비자명 진부분군을 갖는다.

순환군이 아닌 군을 $G$라고 하자. $G$는 항등원 $e$를 제외한 다른 원소를 적어도 하나 가진다. 없다면 $G = \left\{ e \right\} = \langle e \rangle$이기 때문이다.

$a \in G, a \neq e$인 $a$에 대해 $\langle a \rangle \neq G$이다. 

그리고 $G$는 $a$를 포함하는 군이므로, $\langle a \rangle \leq G$이다.

$ \langle a \rangle < G$이므로, $\langle a \rangle$은 $G$의 비자명 진부분군이다. \qed

따라서, 비자명 진부부군을 갖지 않는 군은 순환군이다.


\subsection{6.48.}
전체 군을 $G$라고 하자. 유한개의 부분군 중 순환부분군들에 대해 생각하자. 순환부분군의 개수는 유한개(=$k$개)이고, 각각을 
\begin{align*}
\langle a_1 \rangle, \langle a_2 \rangle, ..., \langle a_k \rangle
\end{align*}라 하자.
\begin{align*}
\bigcup_{i=1}^{k} {\langle a_i \rangle} = G \text{임을 보이자.}
\end{align*}
\textbf{pf.}
$\langle a_i \rangle$은 각각 $G$의 순환부분군이므로, 
\begin{align*}
&\bigcup_{i=1}^{k} {\langle a_i \rangle} \subset G &(1)
\end{align*}

임의의 $x \in G$에 대해 $\langle x \rangle$도 $G$의 순환부분군이므로, 어떤 $1 \le y \le k, y \in \mathbb{N}$가 존재하여 
\begin{align*}
\langle x \rangle = \langle a_y \rangle
\end{align*}
이다. 따라서, $\forall x \in G, \langle x \rangle \subset \bigcup_{i=1}^{k} {\langle a_i \rangle}$이므로
\begin{align*}
&\bigcup_{i=1}^{k} {\langle a_i \rangle} \supset G &(2)
\end{align*}

(1), (2)에 의해 $\bigcup_{i=1}^{k} {\langle a_i \rangle} = G$이다. \qed

\begin{align*}
\forall i \in \left\{1, 2, ..., k\right\}, \langle a_i \rangle \text{가 무한군이 아님을 보이자.}
\end{align*}
\textbf{pf. 귀류법} 어떤 $y \in \left\{1, 2, ..., k\right\}$가 존재하여 $\langle a_y \rangle$가 무한군인 순환부분군이라고 가정하자. $\langle a_y \rangle$의 위수는 무한인 순환군이므로 $\langle \mathbb{Z}, + \rangle$와 동형이고, 이는 무한히 많은 부분수환군을 갖는다. 따라서 $\langle a_y \rangle$은 무한히 많은 부분순환군을 가지며, 이들은 또한 $G$의 부분순환군이다.

하지만, $G$의 부분순환군은 $k$개로 유한하므로 불가능하다. 따라서, $\forall i \in \left\{1, 2, ..., k\right\}, \langle a_i \rangle$은 무한군이 아니다. \qed

\textbf{결론} 따라서, $G$은 $k$개의 위수가 유한한 부분순환군의 합집합이므로 유한군이다.

\subsection{6.52.}
$\mathbb{Z}_n$의 생성원의 개수는 $n$과 서로소인 $1 \le i \le n, i \in \mathbb{N}$의 개수와 같다. 즉, $\phi(n)$이다.

$\phi(p^r) = (p-1) p^{r-1}$이므로, $\mathbb{Z}_{p^r}$의 생성원의 개수는 $(p-1) p^{r-1}$이다.

\subsection{6.55.}
$\forall i \in \mathbb{N}, 1 \le i < p$에 대해 $\text{gcd}(i, p) = 1$이므로, 
\begin{align*}
\langle i \rangle = \langle 1 \rangle = \mathbb{Z}_p
\end{align*}
이다. 따라서, $0 \in \mathbb{Z}_p$을 제외한 원소를 포함하는 군은 $\mathbb{Z}_p$가 될 수 밖에 없다. 그러므로, $p$가 소수이면 $\mathbb{Z}_p$는 비자명 진부분군을 갖지 않는다. 
\subsection{8.12.}
$\left\{ 1, 2, 4, 3 \right\}$
\subsection{8.44.}
$\left|D_n \right| = 2n$이다. 

회전이동들로만 이루어진 치환($\rho_i$)들은 군이며, 위수가 $n$이다.
\subsection{8.45.}
양 면의 중심을 연결하는 직선 기준으로 회전하는 부분군은 위수가 4이며, 직선은 총 3개 존재한다. 따라서, 적어도 위수가 4인 3개의 부분군을 갖는다.

대각선을 기준으로 회전하는 부분군은 위수가 3이며, 대각선은 총 4개 존재한다. 따라서, 적어도 위수가 3인 4개의 부분군을 갖는다.

또한 맞은편 선분의 두 중심을 연결하는 직선 기준으로 회전하는 부분군은 위수가 2이며, 직선은 총 6개 존재한다. 따라서, 적어도 위수가 2인 6개의 부분군을 갖는다.

이 군의 위수는 $(4-1) \times 3 + (3-1) \times 4 + (2-1) \times 6 + 1 = 24$이다.

\subsection{(1)}

$\phi : \mathbb{Z} \rightarrow n\mathbb{Z}$를 $\phi(i) = ni$로 정의하자. 

\textbf{1. $\phi$는 전단사함수이다.}
 왜냐하면
\begin{align*}
&\phi(i) = \phi(j)
\\ \Rightarrow& ni = nj
\\ \Rightarrow& i = j
\end{align*}이므로 단사함수이고,
\begin{align*}
\forall x \in n\mathbb{Z}, \exists k = \frac{x}{n} \in \mathbb{Z} \mid \phi(k) = x
\end{align*}이므로 전사함수이기 때문이다.

\textbf{2. $\forall x, y \in \mathbb{Z} \mid \phi(x + y) = \phi(x) + \phi(y)$이다.}

$\forall x, y \in \mathbb{Z}$에 대하여
\begin{align*}
\phi(x + y) =& n(x + y)
\\ =& nx + ny
\\ =& \phi(x) + \phi(y)
\end{align*}

따라서, $\langle \mathbb{Z}, + \rangle \simeq \langle n\mathbb{Z}, + \rangle$이다.

\subsection{(2)}
$x, y \in V$에 대해 binary operation *을 
\begin{align*}
*(x, y) = \frac{x \cdot y}{2} 
\end{align*}
로 정의하자.

이제 $\langle V, * \rangle$이 Group인지 확인하자.

\textbf{0. 닫혀있다.}
$\forall x, y \in V$에 대해, $x * y \in \mathbb{C}^*$이고
\begin{align*}
\left| x * y \right| =& \left| \frac{x \cdot y}{2} \right|
\\ =& \frac{\left| x \right| \left| y \right| }{\left| 2 \right|}
\\ =& 2
\end{align*}이므로 $V$는 $*$에 의해 닫혀있다.

\textbf{1. 결합법칙 성립}

$\forall a, b, c \in V$에 대하여
\begin{align*}
(a * b) * c =& \frac{a \cdot b}{2} * c
\\ =& \frac{\frac{a \cdot b}{2} \cdot c} {2}
\\ =& \frac{a \cdot \frac{b \cdot c}{2}} {2}
\\ =& a * \frac{b \cdot c}{2}
\\ =& a * (b * c)
\end{align*}가 성립한다.

\textbf{2. 항등원 존재}
$\exists e = 2$에 대해 $\forall x \in Z \mid x * e = e * x = x$이다.

\textbf{3. 역원 존재}
$\forall x \in Z, \exists x^{-1} = \frac{4}{x} \mid x * x^{-1} = x^{-1} * x = e$이다.

따라서, $\langle V, * \rangle$은 Group이다. \qed

\textbf{$\langle U, \cdot \rangle$와 $\langle V, * \rangle$가 동형임을 보이자.}

$\phi : U \rightarrow V$를 $x \in U \mid \phi(x) = 2x$로 정의하자.

\textbf{1. $\phi$는 전단사함수이다.}
왜냐하면
\begin{align*}
&\phi(i) = \phi(j)
\\ \Rightarrow& 2i = 2j
\\ \Rightarrow& i = j
\end{align*}이므로 단사함수이고,
\begin{align*}
\forall x \in V, \exists k = \frac{x}{2} \in U \mid \phi(k) = x
\end{align*}이므로 전사함수이기 때문이다.

\textbf{2. $\forall x, y \in U \mid \phi(x \cdot y) = \phi(x) * \phi(y)$이다.}

$\forall x, y \in U$에 대하여
\begin{align*}
\phi(x \cdot y) =& 2(x \cdot y)
\\ =& \frac{2x \cdot 2y}{2}
\\ =& (2x * 2y)
\\ =& \phi(x) * \phi(y)
\end{align*}이다.

따라서, $\langle U, \cdot \rangle$와 $\langle V, * \rangle$는 동형이다. \qed


\subsection{(3)}
$\forall \sigma \in \text{Aut}(G), \sigma$는 전단사함수이며, $\forall x, y \in G, \phi(x\cdot y) = \phi(x) \cdot \phi(y)$가 성립한다.

\subsubsection{(a) $\langle \text{Aut}(G), \circ \rangle$이 Group인지 확인하자.}
\textbf{0. 닫혀있다.} $\forall x, y \in \text{Aut}(G)$에 대해 $x \circ y$은 두 전단사함수의 합성이므로 전단사함수이다.

또한, $\forall a, b \in G$에 대하여
\begin{align*}
(x \circ y)(a \cdot b) =& x(y(a \cdot b))
\\ =& x(y(a) \cdot y(b))
\\ =& x(y(a)) \cdot x(y(b))
\\ =& (x \circ y) (a) \cdot (x \circ y)(b)
\end{align*}이다. 그러므로, $(x \circ y) \in  \text{Aut}(G)$이다. 

따라서, $\text{Aut}(G)$은 연산 $\circ$에 의해 닫혀있다.

\textbf{1. 결합법칙 성립}
함수의 합성연산은 결합법칙이 성립하므로,  $\text{Aut}(G)$에서 연산 $\circ$의 결합법칙은 성립힌다.

\textbf{2. 항등원 존재} $e : G \rightarrow G \mid x \in G, e(x) = x$로 생각하면, $\forall \sigma \in \text{Aut}(G)$에 대하여
\begin{align*}
e \circ \sigma = \sigma \circ e = \sigma
\end{align*}이다.

\textbf{3. 역원 존재} $\forall \sigma \in G$에 대해 $\sigma$는 전단사함수이므로 역함수 $\sigma ^{-1}$가 존재한다.
\begin{align*}
\sigma^{-1} \circ \sigma = \sigma \circ \sigma^{-1} = e
\end{align*}이다.

따라서, $\langle \text{Aut}(G), \circ \rangle$은 Group이다.


\subsubsection{(b) Find $\text{Aut}(\mathbb{Z}_{12})$}
아래 풀이에서 곱셈은 $\text{mod} 12$로의 곱셈을 의미한다.

$\forall \sigma \in \text{Aut}(\mathbb{Z}_{12})$

$\mathbb{Z}_{12} = \langle 1 \rangle = \left\{n1 \mid n \in \mathbb{Z} \right\}$

By $\sigma$, $\mathbb{Z}_{12} = \langle \sigma(1) \rangle = \left\{n\sigma(1) \mid n \in \mathbb{Z} \right\}$

$\sigma(n1) = n \sigma(1)$이므로, 
\begin{align*}
&\sigma(x) = ax &(a = \sigma(1))
\end{align*}꼴의 함수임을 알 수 있다.

gcd$(a, 12) \neq 1$인 경우에는 $\sigma$가 전사함수가 아니게 되므로, $\text{gcd}(a, 12) = 1$이다. 가능한 $\sigma(1) = a$는 $1, 5, 7, 11$이 있다. 이 $a$들에 대해서는 $\sigma$는 전단사함수이고,
\begin{align*}
\sigma(x +_{12} y) =& a(x+_{12}y)
\\ =& ax +_{12} ay
\\ =& \sigma(x) +_{12} \sigma(y)
\end{align*}이므로 가능한 동형사상이다.

따라서, 
\begin{align*}
\text{Aut}(\mathbb{Z}_{12}) = 
\{ &\sigma_{1} : \mathbb{Z}_{12} \rightarrow \mathbb{Z}_{12}  \mid \sigma_{1}(x) = x, 
\\ &\sigma_2 : \mathbb{Z}_{12} \rightarrow \mathbb{Z}_{12} \mid \sigma_2(x) = 5x, 
\\ &\sigma_3 : \mathbb{Z}_{12} \rightarrow \mathbb{Z}_{12} \mid \sigma_3(x) = 7x, 
\\ &\sigma_4 : \mathbb{Z}_{12} \rightarrow \mathbb{Z}_{12} \mid \sigma_4(x) = 11x  \}
\end{align*}
 

\end{document}










































