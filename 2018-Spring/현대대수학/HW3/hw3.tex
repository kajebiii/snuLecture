%%%%%%%%%%%%%%%%%%%%%%%%%%%%%%%%%%%%%%%%%
% Programming/Coding Assignment
% LaTeX Template
%
% This template has been downloaded from:
% http://www.latextemplates.com
%
% Original author:
% Ted Pavlic (http://www.tedpavlic.com)
%
% Note:
% The \lipsum[#] commands throughout this template generate dummy text
% to fill the template out. These commands should all be removed when 
% writing assignment content.
%
% This template uses a Perl script as an example snippet of code, most other
% languages are also usable. Configure them in the "CODE INCLUSION 
% CONFIGURATION" section.r
%
%%%%%%%%%%%%%%%%%%%%%%%%%%%%%%%%%%%%%%%%%

%----------------------------------------------------------------------------------------
%	PACKAGES AND OTHER DOCUMENT CONFIGURATIONS
%----------------------------------------------------------------------------------------

\documentclass{article}
\usepackage[hangul]{kotex}
\usepackage{fancyhdr} % Required for custom headers
\usepackage{lastpage} % Required to determine the last page for the footer
\usepackage{extramarks} % Required for headers and footers
\usepackage[usenames,dvipsnames]{color} % Required for custom colors
\usepackage{graphicx} % Required to insert images
\usepackage{listings} % Required for insertion of code
\usepackage{courier} % Required for the courier font
\usepackage{lipsum} % Used for inserting dummy 'Lorem ipsum' text into the template
\usepackage{amsthm,amsmath}
\usepackage[table,xcdraw]{xcolor}

\usepackage{verbatim} % Required for multiple comment
%\usepackage{MnSymbol} % cupdot
\usepackage{amsmath} % Required for use \therefore \because and others..
\usepackage{amssymb} % Required for use \therefore \because and others..
\usepackage{algorithm, algpseudocode}
\usepackage{verbatim} % for commment, verbatim environment
\usepackage{spverbatim} % automatic linebreak verbatim environment
\usepackage{listings}
\usepackage{ulem}
\usepackage{hyperref}
\usepackage{datetime} % Used for showing version as last modified time
\yyyymmdddate
\DeclareGraphicsExtensions{.pdf,.png,.jpg}

\usepackage{xcolor}




% Margins
\linespread{1.25} % Line spacing
\usepackage[a4paper,top=2cm,bottom=1cm,left=1cm,right=1cm,marginparwidth=1.75cm]{geometry}
\usepackage{titlesec,sectsty}
\setlength\parindent{0pt}
\setlength\parskip{10pt}
\setlength{\abovedisplayskip}{3pt}
\setlength{\belowdisplayskip}{3pt}
\sectionfont{\fontsize{12}{10}\selectfont}
\subsectionfont{\fontsize{10}{10}\selectfont}
\titlespacing{\subsection}{0pt}{-\parskip}{-\parskip}
\usepackage{multicol}
\usepackage{cuted}
\newenvironment{Figure}
{\par\medskip\noindent\minipage{\linewidth}}
{\endminipage\par\medskip}


% mathematics
\usepackage{amsmath,amssymb,mathtools}
\usepackage{cancel} % cancelation of terms
\usepackage{nicefrac} % certain fraction styles
\newcommand*\diff{\mathop{}\!\mathrm{d}} % differential
\DeclareMathOperator{\arsinh}{arsinh}

% Set up the header and footer
\pagestyle{fancy}
\lhead{JongBeom Kim}
\chead{현대대수학 HW 3} % Top center head
\rhead{2018 Spring} % Top right header
\lfoot{\lastxmark} % Bottom left footer
\cfoot{} % Bottom center footer
\renewcommand\headrulewidth{0.4pt} % Size of the header rule
\renewcommand\footrulewidth{0.4pt} % Size of the footer rule
\newcommand{\myul}[2][black]{\setulcolor{#1}\ul{#2}\setulcolor{black}}


\setlength\parindent{0pt} % Removes all indentation from paragraphs

%----------------------------------------------------------------------------------------
%	CODE INCLUSION CONFIGURATION
%----------------------------------------------------------------------------------------


\setcounter{secnumdepth}{0} % Removes default section numbers
\newcounter{homeworkProblemCounter} % Creates a counter to keep track of the number of problems


\renewcommand\headrule
{
	\begin{minipage}{1\textwidth}
		\hrule width \hsize height 1pt \kern 1.5pt \hrule width \hsize height 0.5pt  
	\end{minipage}\par
}%

% lists
\usepackage{enumitem}
\setlist[itemize]{topsep=-10pt} %set spacing for itemize
\setitemize{itemsep=3pt}
\setlist[enumerate]{topsep=-10pt} %set spacing for itemize
\setenumerate{itemsep=3pt}

%----------------------------------------------------------------------------------------
%	TITLE PAGE
%----------------------------------------------------------------------------------------

%----------------------------------------------------------------------------------------



\begin{document}
\twocolumn


\chapter{Homework 4}

\section{18.40.}
\textbf{귀류법.} $2\mathbb{Z}$와 $3\mathbb{Z}$가 환으로서 동형이라고 하자.
그러면 동형사상
$$\phi : 2\mathbb{Z} \rightarrow 3\mathbb{Z}$$
가 존재한다. 이 함수는 일대일이고 전사이다.

$\phi(2) = a$라 하면, $\phi(a + b) = \phi(a) + \phi(b)$이므로, 
$$\phi(2n) = an$$이다. 이 함수는 전사이므로, 
$$3 \bigg| \left| a\right|$$이고, 이는 $a = -3, 3$임을 의미한다.

\subsection{$a=-3$인 경우}
$$-6 = \phi(2) + \phi(2) = \phi(4) = \phi(2) \phi(2) = 9$$
이므로 모순이다.
\subsection{$a=3$인 경우}
$$6 = \phi(2) + \phi(2) = \phi(4) = \phi(2) \phi(2) = 9$$
이므로 모순이다.

따라서, $2\mathbb{Z}$와 $3\mathbb{Z}$가 환으로서 동형이 아니다. \qed

\section{18.52.}
예 18.15.에 의해서, $\mathbb{Z}_{rs}$, $\mathbb{Z}_r \times \mathbb{Z}_s$사이에 동형을 주는 사상
$$\phi : \mathbb{Z}_{rs} \rightarrow \mathbb{Z}_r \times \mathbb{Z}_s$$
가 존재한다. 여기서 $\phi$는
$$\phi(x) = (x) \cdot (1, 1)$$이다.

나눗셈 알고리즘을 이용하여 
$$m = m_0 r + m_1 \quad (0 \le m_1 < r)$$ 
$$n = n_0 s + n_1 \quad (0 \le n_1 < s)$$라 하자. 
그러면 $(m_1, n_1) \in \mathbb{Z}_r \times \mathbb{Z}_s$이다.

$\phi$가 동형사상이므로 전사이면서 단사이다. 따라서,
$$\exists x \in \mathbb{Z}_{rs} \bigg| \phi(x) = (m_1, n_1)$$
이고, 이는 
$$x \equiv m_1 \equiv m \mod r$$
$$x \equiv n_1 \equiv n \mod s$$
을 의미한다.

따라서, $\forall m, n \in \mathbb{Z}$에 대해 
$x \equiv m \mod r$와
$x \equiv n \mod s$을 만족하는 정수 $x$가 존재한다. \qed

\section{18.55.}
$$2a = (a + a) = (a + a) ^ 2 = 4a^2 = 4a$$
이므로,
$$a = -a$$이다. 또한
$$a+b = (a+b)^2 = a^2 + ab + ba + b^2 = a + ab + ba + b$$
이므로 $a = -a$을 이용하면,
$$ab = -ba = (-b)a = ba$$이다.

따라서, 모든 부울환은 가환환이다.

\section{19.23.}
나눗셈환이므로 $0, 1 \in R$이다. $(\text{단}, 0 \neq 1)$
또한, $$0^2 = 0$$ $$1^2 = 1$$이므로, $0, 1$은 멱등원이다.

$(a \neq 0) \wedge (a \neq 1)$이면 $a$가 멱등원이 아님을 보이자.

\textbf{pf.} 나눗셈환이므로 $a$는 가역원이고, 
$$\exists b \in R \bigg| ab = ba = 1$$이다. 만약 $$a^2 = a$$라 하면, 이 식에 $b$를 오른쪽에 곱하게 되면
$$aab = ab \iff a = 1$$
이므로, 모순이다. 

따라서 $(a \neq 0) \wedge (a \neq 1)$이면 $a$는 멱등원이 아니고, $R$의 멱등원은 $0, 1$ 두개 뿐이다. \qed

\section{19.29.}
정역 $D$의 표수가 $0$이 아니라면 소수임을 보이면 충분하다.

정리 19.15.에 의해 표수가 0이 아니면, 표수는 $n \cdot 1 = 0$을 만족하는 가장 작은 자연수 $n$이다.

이 자연수 $n$이 소수가 아니라고 하자. 즉, $n = pq \quad (2 \le p, q \in \mathbb{Z}^+)$라 하자. 그러면 
$$0 = n \cdot 1 = (p \cdot 1) (q \cdot 1)$$이고, $D$는 정역이므로 $0$의 약수가 존재하지 않는다. 따라서, $$(p \cdot 1 = 0) \vee (q \cdot 1 = 0)$$라 할 수 있다. $p, q < n$이므로, 이는 $n$이 $n \cdot 1 = 0$을 만족하는 가장 작은 자연수임에 모순이다.

따라서, $n$은 소수일 수 밖에 없다. \qed

\section{20.6.}
$$2^{17} \equiv 0 \mod 2$$
$$2^{17} \equiv (2^6)^2 (2^5) \equiv 5 \mod 9$$
이므로
$$2^{17} \equiv 14 \mod 18$$
이다. 이를 이용하면,
\begin{align*}
2^{2^{17}} + 1 \equiv& 2^{14} + 1 \equiv (2^4)^3 (2^2) + 1 \equiv (-3)^3 (2^2) + 1
\\ \equiv& -108 + 1 \equiv -13 + 1 \equiv 7 \mod 19
\end{align*}
이다.

\section{20.14.}
\begin{align*}
(45x \equiv 15 \mod 24) \iff& (-3x \equiv 15 \mod 24)
\\ \iff&  (3x \equiv -15 \mod 24)
\\ \iff&  (3x \equiv 9 \mod 24)
\\ \implies& (x \equiv 3 \mod 8)
\end{align*}
이고, $8 \cdot 3 = 24$이므로, 따라서 주어진 합동식의 모든 해는
$$(3 + \mathbb{Z}_{24})$$
$$(11 + \mathbb{Z}_{24})$$
$$(19 + \mathbb{Z}_{24})$$
에 속하는 경우이다.

\section{20.27.}
자기 자신이 자신의 곱셉역원이 된다는 말은
$$x^2 = 1$$
라는 것이다.

$\mathbb{Z}_p$는 정역이므로, 
$$x^2 = 1 \iff (x-1)(x+1) = 0$$
이면 $0$의 약수가 존재하지 않으므로,
$$(x-1 = 0) \vee (x+1 = 0)$$이다.

즉, $\mathbb{Z}_p$의 원소중에서는 $x^2 = 1$을 만족하는 $x$는 $1, p-1$뿐이다. 

\section{20.28.}
\subsection{$p=2$인 경우}
$$(p-1)! \equiv (1)! \equiv 1 \equiv = -1  \mod p$$
이므로 성립한다.
\subsection{$p\neq2$인 경우}
먼저, $1 \neq p-1$이다. $1$의 역원은 $1$이고, $p-1$의 역원은 $p-1$임을 문제 20.27.에서 보였다. 

$a$와 $b$가 다르면, $a$의 역원과 $b$의 역원도 다르므로, 
$$(x \neq 1) \vee (x \neq p-1) \vee (0 \neq x \in \mathbb{Z}_p)$$
이면 $x$의 역원은 $1$과 $p-1$이 아니다.

또, 문제 20.27.에 의해 $2, 3, \cdots, p-2$은 자기자신이 자신의 역원이 아니므로, $2, 3, \cdots, p-2$에서 서로 역원이 되게 하는 $\dfrac{p-3}{2}$개의 쌍을 만들어 줄 수 있다.

이를 이욯하여 $(p-1)!$을 생각하면,
\begin{align*}
(p-1)! \equiv& (1) \left((2)(3) \cdots (p-2) \right) (p-1)
\\ \equiv& (1) \left( (1)(1) \cdots (1) \right) (p-1) \tag{단, 가운데 중괄호 안의 $1$의 개수는 $\dfrac{p-3}{2}$개이다.}
\\ \equiv& (p-1) \equiv -1 \mod p
\end{align*}
임을 알 수 있다.

따라서, 모든 소수 $p$에 대해 $(p-1)! \equiv -1 \mod p$이다. \qed

\section{20.29.}
\subsection{Lemma. 1}
$a$가 정수, $b$가 양의 정수, $p$가 소수이고, $p-1 \mid b-1$이면
$$a^b - a \equiv 0 \mod p$$
임을 보이자.
\textbf{pf.}

모든 $b = n(p-1) + 1$에 대해 보이면 충분하자. 이를 수학적 귀납법으로 보이자.

$n = 1$인 경우는 페르마 소정리에 의해 $a^{p} - a \equiv 0 \mod p$이다. 

$n = k$인 경우에 성립한다고 가정하자. 그러면, $b = (k+1)(p-1) + 1$일 때를 생각하면
\begin{align*}
a^{b} - a \equiv& a^{p} a^{b-p} - a \equiv (a)(a^{b-p}) - a \equiv a^{b-(p-1)} - a
\\ \equiv& a^{(k)(p-1) + 1} - a \equiv 0 \mod p
\end{align*}
이다. 따라서, $n = k+1$인 경우에도 성립한다.

따라서, 수학적 귀납법에 의해 모든 자연수 $n$에 대해 $b = n(p-1) + 1$이면 
$$a^b - a \equiv 0 \mod p$$
이다. \qed

위 Lemma를 이용하면,
$$37-1 \mid 37 - 1 \implies n^{37} - n \equiv 0 \mod 37$$
$$19-1 \mid 37 - 1 \implies n^{37} - n \equiv 0 \mod 19$$
$$13-1 \mid 37 - 1 \implies n^{37} - n \equiv 0 \mod 13$$
$$7-1 \mid 37 - 1 \implies n^{37} - n \equiv 0 \mod 7$$
$$3-1 \mid 37 - 1 \implies n^{37} - n \equiv 0 \mod 3$$
$$2-1 \mid 37 - 1 \implies n^{37} - n \equiv 0 \mod 2$$
이다. 따라서,
$$n^{37} - n \equiv 0 \mod 383838$$
임을 알 수 있다.

\section{22.11.}
문제 20.29.의 Lemma를 이용하면,
$$3x^{106}+5x^{99}+2x^{53} \equiv 3x^4 + 5x^3 + 2x^5 \mod 7$$
이므로,
\begin{align*}
\phi_4(3x^{106}+5x^{99}+2x^{53}) \equiv& (3)4^4 + (5)4^3 + (2)4^5
\\ \equiv& (3)(2)^2 + (5)(2)(4) + (2)(2)^2(4)
\\ \equiv& 12 + 40 + 32 \equiv 84 \equiv 0 \mod 7
\end{align*}
이다. 따라서 $0$이다.

\section{22.17.}
문제 20.29.의 Lemma를 이용하면,
\begin{align*}
2x^{219}+3x^{74}+2x^{57}+3x^{44} \equiv& 2x^{3}+3x^{2}+2x+3x^{4} \mod 5
\\  \equiv& 2x+3x^{2}+2x^{3}+3x^{4} \mod 5
\end{align*}
이다.

$$2x+3x^{2}+2x^{3}+3x^{4} = (x)(3x+2)(x^2 + 1)$$

모든 $x \in \mathbb{Z}_5$에 대해 대입해보면,
$$\phi_0(x) = 0 \implies \phi_0(2x+3x^{2}+2x^{3}+3x^{4}) = 0$$
$$\phi_1(3x+2) = 0 \implies \phi_1(2x+3x^{2}+2x^{3}+3x^{4}) = 0$$
$$\phi_2(x^2+1) = 0 \implies \phi_2(2x+3x^{2}+2x^{3}+3x^{4}) = 0$$
$$\phi_3(x^2+1) = 0 \implies \phi_3(2x+3x^{2}+2x^{3}+3x^{4}) = 0$$
$$(4)(12+2)(16+1) \equiv 2 \mod 5 \implies \phi_4(2x+3x^{2}+2x^{3}+3x^{4}) = 2$$
이다. 따라서, 해는 
$$\left\{0, 1, 2, 3\right\}$$
이다.

\section{22.22.}
$x^4 = x^2 \mod 4$이다. 따라서, $3$차 이하 다항식만 생각해도 충분하다. 프로그래밍을 통해 구한 결과는 다음과 같다. (상수항이 둘다 1인 경우만 생각하였다. 둘다 3인 경우는 각각에 3을 곱해주면 된다.)
$$(0x^3+0x^2+2x+1)(0x^3+0x^2+2x+1) \equiv 1 \mod 4$$
$$(3x^3+0x^2+1x+1)(1x^3+0x^2+3x+1) \equiv 1 \mod 4$$
$$(3x^3+0x^2+3x+1)(1x^3+0x^2+1x+1) \equiv 1 \mod 4$$
$$(2x^3+0x^2+0x+1)(2x^3+0x^2+0x+1) \equiv 1 \mod 4$$
$$(2x^3+0x^2+2x+1)(2x^3+0x^2+2x+1) \equiv 1 \mod 4$$
$$(1x^3+0x^2+1x+1)(3x^3+0x^2+3x+1) \equiv 1 \mod 4$$
$$(1x^3+0x^2+3x+1)(3x^3+0x^2+1x+1) \equiv 1 \mod 4$$
$$(2x^3+1x^2+1x+1)(0x^3+1x^2+3x+1) \equiv 1 \mod 4$$
$$(2x^3+1x^2+3x+1)(0x^3+1x^2+1x+1) \equiv 1 \mod 4$$
$$(1x^3+1x^2+0x+1)(1x^3+1x^2+0x+1) \equiv 1 \mod 4$$
$$(1x^3+1x^2+2x+1)(1x^3+1x^2+2x+1) \equiv 1 \mod 4$$
$$(0x^3+1x^2+1x+1)(2x^3+1x^2+3x+1) \equiv 1 \mod 4$$
$$(0x^3+1x^2+3x+1)(2x^3+1x^2+1x+1) \equiv 1 \mod 4$$
$$(3x^3+1x^2+0x+1)(3x^3+1x^2+0x+1) \equiv 1 \mod 4$$
$$(3x^3+1x^2+2x+1)(3x^3+1x^2+2x+1) \equiv 1 \mod 4$$
$$(0x^3+2x^2+0x+1)(0x^3+2x^2+0x+1) \equiv 1 \mod 4$$
$$(0x^3+2x^2+2x+1)(0x^3+2x^2+2x+1) \equiv 1 \mod 4$$
$$(3x^3+2x^2+1x+1)(1x^3+2x^2+3x+1) \equiv 1 \mod 4$$
$$(3x^3+2x^2+3x+1)(1x^3+2x^2+1x+1) \equiv 1 \mod 4$$
$$(2x^3+2x^2+0x+1)(2x^3+2x^2+0x+1) \equiv 1 \mod 4$$
$$(2x^3+2x^2+2x+1)(2x^3+2x^2+2x+1) \equiv 1 \mod 4$$
$$(1x^3+2x^2+1x+1)(3x^3+2x^2+3x+1) \equiv 1 \mod 4$$
$$(1x^3+2x^2+3x+1)(3x^3+2x^2+1x+1) \equiv 1 \mod 4$$
$$(2x^3+3x^2+1x+1)(0x^3+3x^2+3x+1) \equiv 1 \mod 4$$
$$(2x^3+3x^2+3x+1)(0x^3+3x^2+1x+1) \equiv 1 \mod 4$$
$$(1x^3+3x^2+0x+1)(1x^3+3x^2+0x+1) \equiv 1 \mod 4$$
$$(1x^3+3x^2+2x+1)(1x^3+3x^2+2x+1) \equiv 1 \mod 4$$
$$(0x^3+3x^2+1x+1)(2x^3+3x^2+3x+1) \equiv 1 \mod 4$$
$$(0x^3+3x^2+3x+1)(2x^3+3x^2+1x+1) \equiv 1 \mod 4$$
$$(3x^3+3x^2+0x+1)(3x^3+3x^2+0x+1) \equiv 1 \mod 4$$
$$(3x^3+3x^2+2x+1)(3x^3+3x^2+2x+1) \equiv 1 \mod 4$$

$$(2x+1)^2 \equiv 1 \mod 4$$가 가장 간단한 다항식인 것 같다.

\section{22.25.}
\subsection{a.}
$f(x), g(x) \in D[x]$이고, $f(x) \neq 0, g(x) \neq 0$라 하자.

$\text{deg} f(x) = n$라 하고, $\text{deg} g(x) = m$라 하면,
$$\text{deg} (f(x)g(x)) = n+m$$이다.

가역원이기 위해서는 $n+m = 0$이므로, $n = m = 0$이다.

차수가 $0$일때에는, $D$의 가역원이 $D[x]$에서 가역원이며 $D$의 가역원이 아닌 원소는 $D[x]$에서도 가역원이 아니다.

따라서, $D[x]$의 가역원은 $D$의 가역원이다.
\subsection{b.}
a.에 의해 $1, -1$뿐이다.
\subsection{c.}
a.에 의해 $1, 2, 3, 4, 5, 6$뿐이다.

\section{23.7.}
가역원은
$$\left\{1, 2, 3, 4, 5, 6, 7, 8, 9, 10, 11, 12, 13, 14, 15, 16\right\}$$
이고, 여기서 $3$이 가역원들의 곱셈순환군을 생성한다.

6.16.의 정리를 이용하면, 생성원은
$$\left\{3^1, 3^3, 3^5, 3^7, 3^9, 3^11, 3^13, 3^15\right\}$$
$$\left\{3, 10, 5, 11, 14, 7, 12, 6\right\}$$
임을 알 수 있다.

\section{23.9.}
\begin{align*}
x^4 + 4 \equiv& x^4 - 1 \mod 5
\\ \equiv& (x^2 - 1)(x^2 + 1) \mod 5
\\ \equiv& (x^2 - 1)(x^2 - 4) \mod 5
\\ \equiv& (x-1)(x+1)(x-2)(x+2) \mod 5
\\ \equiv& (x+1)(x+2)(x+3)(x+4) \mod 5
\end{align*}
이다.
\section{23.17.}
$$f(x) = x^4 - 22 x^2 + 1$$
라 하자. 만약 $f(x)$가 $\mathbb{Q}[x]$에서 일차인수를 가지면, $f(x)$는 유리수해를 가져야한다. 그러면 정리 23.12.에 의해 $f(x)$는 정수해 $m$을 가지고 $m$이 $1$의 약수가 되는데, $f(1) = f(-1) = -20$이므로 $f(x)$는 $\mathbb{Q}[x]$에서 일차인수를 가지지 않는다.

이제 만약 $f(x)$가 $\mathbb{Q}[x]$에서 인수분해가 된다면, 두 이차 다항식의 곱으로 될 수 밖에 없다. 여기서 정리 23.11.을 이용하면 $\mathbb{Z}[x]$에서 두 이차 다항식의 곱으로 인수분해가 되고, 그러면 $a, b, c, d \in \mathbb{Z}$에 대해,
$$f(x) = x^4 - 22 x^2 + 1 = (x^2+ax+b)(x^2+cx+d)$$
라 할 수 있다. 그러면,
$$a+c = 0, \quad b + d + ac = -22, \quad ad + bc = 0, \quad bd = 1$$
을 만족해야하고, 
$$(a + c = 0)$$이면서, $$( (b = d = 1) \vee (b = d - 1) $$이므로,
$$(a^2 = 24) \vee (a^2 = 20)$$인데 이는 불가능하다.

따라서, $f(x)$는 $\mathbb{Q}[x]$에서 기약이다. \qed

\section{23.28.}
정리 23.10.에 의해 차수가 3인 다항식이 기약이기 위한 필요충분조건은 $\mathbb{Z}_2$에서 해를 가지지 않는 것임을 안다. 즉, 3차 다항식 $f(x)$를
$$f(x) = x^3 + ax^2 + bx + c$$
라 하면, $$f(0) != 0, \quad f(1) != 0$$이므로,
$$c = 1, \quad 1 + a + b + c \equiv 1 \mod 2$$이다.

즉, $c = 1, \quad a + 1\equiv b \mod 2$이다.

따라서, 가능한 모든 $\mathbb{Z}_2[x]$의 3차 기약 다항식은
$$\left\{x^3+x^2+1, x^3+x+1\right\}$$
이다.
\section{23.34.}
$$f(x) = x^p + a$$라 하면,
$$f(-a) \equiv (-a)^p + a \equiv -a + a \equiv 0 \mod p$$이므로,
$f(x)$는 $\mathbb{Z}_p[x]$에서 기약이 아니다.

\section{23.35.}
$a \neq 0$이 $f(x)$의 해이므로,
$$0 = f(a) = a_0 + a_1 a + a_2 a^2 + \cdots + a_n a^n$$이다. 이 식의 양변을 $a^n$으로 나누면,
$$0 = a_n + a_{n-1}(\frac{1}{a}) + a_{n-2}(\frac{1}{a})^2 + \cdots + a_0(\frac{1}{a})^n$$이므로, $\dfrac{1}{a}$는 $a_n + a_{n-1} x + \cdots a_0 x^n$의 해이다.

\section{26.2.}
$\mathbb{Z}_n$의 $\mathbb{Z}_2$와 동형인 부분환이 있다고 가정하고 $H$라 하자.

덧셈연산에 대해서는 군이므로, 항등원 $0 \in H$이다. 이제 $0$이 아닌 원소 $a \in H$라 하자.
$$a + a = 0, \quad a^2 \equiv a \mod n$$을 만족해야하므로,
$$2 \mid n, \quad a = \frac{n}{2}$$이어야 한다.
$n = 2a$라 하고, $a = 2k + r \quad (r = 0 \text{ or } 1, k \in \mathbb{Z})$라 하면
$$4k^2 + 4rk + r^2 \equiv 2k + r \mod 4k + 2r$$ 
$$2rk \equiv 2k \mod 4k + 2r$$ 
이므로 $r = 1$일 수 밖에 없다. 즉, 
$$n = 2a = 4k + 2$$인 경우에만 $\mathbb{Z}_2$와 동형인 부분환이 존재한다.

\section{26.18.}
체 $F$에서 환 $R$로 가는 준동형사상
$$\phi : F \rightarrow R$$의 Kernel $N = Ker(\phi)$를 생각하자. 일단 $0 \in N$이다.

\subsection{$\left\{0\right\} = N$인 경우}
정리 26.6.에 의해 $\phi$는 일대일 사상이다.

\subsection{$\left\{0\right\} \neq N$인 경우}
$0$이 아닌 원소 $a \in F$가 $a \in N$이다.

$a \neq 0$이고, $F$의 원소이므로 곱셈에 대한 역원 $a^{-1}$가 존재하고
$$\phi(1) = \phi(a)\phi(a^{-1}) = 0$$이므로, $1 \in N$이다.

따라서, 임의의 원소 $f \in F$에 대해 
$$\phi(f) = \phi(f) \phi(1) = 0$$이므로, $N = F$이고 이는 $\phi$가 모든 원소를 $0$으로 보내는 사싱임을 의미한다.

\textbf{결론.} 따라서 체에서 환으로의 모든 준동형사상은 일대일 사상이거나, 모든 원소를 $0$으로 보내느 사상이다. \qed


\section{26.20.}
\subsection{$\phi_p(a+b)  = \phi_p(a) + \phi_p(b)$을 보이자.}
\begin{align*}
\phi_p(a+b) =& (a+b)^p
\\ =& \sum_{i=0}^{p} {\binom{p}{i} \cdot (a^i b^{p-i})} \tag{$R$은 가환환}
\\ =& a^p + b^p \tag{$1 \le i \le p-1 \implies p \bigg| \dbinom{p}{i}$}
\\ =& \phi_p(a) + \phi_p(b)
\end{align*}
\subsection{$\phi_p(ab)  = \phi_p(a) \phi_p(b)$을 보이자.}
\begin{align*}
\phi_p(ab) =& (ab)^p
\\ =& a^p b^p \tag{$R$은 가환환}
\\ =& \phi_p(a)  \phi_p(b)
\end{align*}

따라서, $\phi_p(a) = a^p$로 정의하면 $\phi$는 준동형사상이다.

\section{26.37.}
\subsection{$\phi(a+bi+c+di)  = \phi(a+bi) + \phi(c+di)$을 보이자.}
\begin{align*}
\phi(a+bi+c+di) =& \begin{bmatrix}
a+c& b+d \\ 
-b-d &a+c 
\end{bmatrix}
\\ =& \begin{bmatrix}
a& b \\ 
-b &a 
\end{bmatrix} + \begin{bmatrix}
c& d \\ 
-d &c 
\end{bmatrix}
\\ =& \phi(a+bi) + \phi(c+di)
\end{align*}
\subsection{$\phi((a+bi)(c+di))  = \phi(a+bi) \phi(c+di)$을 보이자.}
\begin{align*}
\phi((a+bi)(c+di)) =& \phi(ac-bd+(ad+bc)i)
\\ =& \begin{bmatrix}
ac-bd& ad+bc \\ 
-ad-bc &ac-bd
\end{bmatrix}
\\ =& \begin{bmatrix}
a& b \\ 
-b &a 
\end{bmatrix} \begin{bmatrix}
c& d \\ 
-d &c 
\end{bmatrix}
\\ =& \phi(a+bi)  \phi(c+di)
\end{align*}

$\phi$는 준동형사상이다. 

\subsection{$Ker(\phi) = \left\{0\right\}$임을 보이자.}
$$\phi(a+bi) = \begin{bmatrix}
0& 0 \\ 
0 &0 
\end{bmatrix}$$이기 위한 조건은
$$a = b = 0$$이므로, 
$Ker(\phi) = \left\{0\right\}$이다. 따라서, $\phi$는 단사함수이다.

$$\phi : \mathbb{C} \rightarrow \phi[\mathbb{C}]$$은 $\phi[\mathbb{C}]$의 정의에 의해 전사함수이다.

위 세가지 증명에 의해서, $\phi$는 $\mathbb{C}$와 $\phi[\mathbb{C}]$는 동형사상을 유도한다.


\section{27.2.}
먼저 정리 19.11.에 의해 유한 정역은 체임을 안다.

또한, 곱셈항등원을 가진 가환환 $R$에서 $M$이 극대 아이디얼이기 위한 필요충분조건은 $R\setminus M$이 체이고, $N$이 소 아이디얼이기 위한 필요충분조건이 $R\setminus N$이 정역이므로, $\mathbb{Z}_{12}$의 소 아이디얼과 극대 아이디얼은 같음을 알 수 있다.

$\mathbb{Z}_{12}$의 진 아이디얼은
$$\left\{0, 1, 2, 3, 4, 5, 6, 7, 8, 9, 10, 11\right\}$$
$$\left\{0, 2, 4, 6, 8, 10\right\}$$
$$\left\{0, 3, 6, 9\right\}$$
$$\left\{0, 4, 8\right\}$$
$$\left\{0, 6\right\}$$
$$\left\{0\right\}$$
이고, 여기서 극대 아이디얼의 성질을 만족하는 것은
$$\left\{0, 2, 4, 6, 8, 10\right\}$$
$$\left\{0, 3, 6, 9\right\}$$
뿐이다.

\section{27.6.}
$\mathbb{Z}_3 [x] \setminus \langle x^3 + x^2 + c \rangle$가 체가 되는 것과 $\langle x^3 + x^2 + c \rangle$가 극대 아이디얼인 것과 동치이다. 또한, $\langle x^3 + x^2 + c \rangle$가 극대 아이디얼인 것은 $x^3 + x^2 + c$가 $\mathbb{Z}_3$위에서 기약인 것과 동치이다.

$f(x) = x^3 + x^2 + c$은 3차 다항식이므로 $\mathbb{Z}_3$에서 해를 가지지 않으면 기약이다.
$$f(0) \equiv c \mod 3$$
$$f(1) \equiv c+2 \mod 3$$
$$f(2) \equiv c \mod 3$$
이므로, $c = 2 \in \mathbb{Z}_3$이다.


\section{27.26.}
$$\mathbb{Z}_2 \times \mathbb{Z}_3$$
은 환이며, 부분환으로
$$\mathbb{Z}_2 \times \left\{0\right\} \simeq \mathbb{Z}_2$$
$$\left\{0\right\} \times \mathbb{Z}_3 \simeq \mathbb{Z}_3$$
을 가지고 있다.

따라서 가능하다.

\section{27.30.}
정리 27.24.에 의해 $F[x]$의 모든 아이디얼은 주 아이디얼임을 상기하자.

임의의 비자명 진 소 아이디얼 $N$을 생각하자. $N = \langle f(x) \rangle$이고, 비자명 아이디얼이므로 $N \neq \left\{0\right\}$이다.

\subsection{$f(x)$는 $F$에서 기약이다.}
만약 $f(x)$가 $F$에서 기약이 아니라고 하면, $f(x)$보다 차수가 낮은 두 다항식 $g(x), h(x) \in F[x]$가 존재하여
$$f(x) = g(x) h(x)$$인데, $g(x) h(x) \in N$이고, $g(x) \notin N, h(x) \notin N$이므로 소 아이디얼임에 모순이다. \qed

또한 정리 27.25.에 의해 아이디얼 $\langle f(x) \rangle$가 극대 아이디얼이 될 필요충분조건이 $f(x)$가 $F$ 위에서 기약임을 알고 있다.

따라서, $F$가 체이면 다항식환 $F[x]$의 모든 비자명 진 소 아이디얼은 극대 아이디얼이다.

\section{27.38.}
$N$이 $M_2 ( \mathbb{Z}_2)$의 비자명 아이디얼일때, $N = M_2 ( \mathbb{Z}_2)$임을 보이면 충분하다.

일단 $O \in N$이다. (덧셈군의 항등원)

$N$이 비자명 아이디얼이므로, $A \neq O$인 $A \in N$을 만족하는 행렬
$$ A = \begin{bmatrix}
a&b  \\ 
c&d 
\end{bmatrix}$$을 생각하자.

아이디얼의 성질에 의해, 
$$ \begin{bmatrix}
0&1  \\ 
1&0 
\end{bmatrix}
A = \begin{bmatrix}
c&d  \\ 
a&b 
\end{bmatrix}
\quad
A\begin{bmatrix}
	0&1  \\ 
	1&0 
\end{bmatrix}
= \begin{bmatrix}
	b&a  \\ 
	d&c 
\end{bmatrix}
$$도 $N$에 속한다.

\subsection{Case 1. $A$에 1이 한개} 
위 성질에 의해
$$\begin{bmatrix}
1&0  \\ 
0&0
\end{bmatrix},\begin{bmatrix}
0&1  \\ 
0&0
\end{bmatrix},\begin{bmatrix}
0&0  \\ 
1&0
\end{bmatrix},\begin{bmatrix}
0&0  \\ 
0&1
\end{bmatrix} \in N$$이다.

$N$이 덧셈부분군이므로,
$$
\begin{bmatrix}
x&y  \\ 
z&w 
\end{bmatrix} = x\begin{bmatrix}
1&0  \\ 
0&0
\end{bmatrix}+y\begin{bmatrix}
0&1  \\ 
0&0
\end{bmatrix}+z\begin{bmatrix}
0&0  \\ 
1&0
\end{bmatrix}+w\begin{bmatrix}
0&0  \\ 
0&1
\end{bmatrix} \in N
$$이다. 따라서, $N = M_2 ( \mathbb{Z}_2)$이다.

\subsection{Case 2. $A$에 1이 두개}
\subsubsection{Case 2-1. $1$이 가로나 세로로 붙어있는 경우}
$$
\begin{bmatrix}
1&0  \\ 
1&0
\end{bmatrix} \in N \implies
\begin{bmatrix}
1&0  \\ 
0&0
\end{bmatrix} 
\begin{bmatrix}
1&0  \\ 
1&0
\end{bmatrix} =
\begin{bmatrix}
1&0  \\ 
0&0
\end{bmatrix} \in N
$$이므로, Case 1.에 의해 $N = M_2 ( \mathbb{Z}_2)$이다.

\subsubsection{Case 2-2. $1$이 대각선에 위치한 경우}
$$
\begin{bmatrix}
1&0  \\ 
0&1
\end{bmatrix} \in N \implies
\begin{bmatrix}
1&0  \\ 
0&0
\end{bmatrix} 
\begin{bmatrix}
1&0  \\ 
0&1
\end{bmatrix} =
\begin{bmatrix}
1&0  \\ 
0&0
\end{bmatrix} \in N
$$이므로, Case 1.에 의해 $N = M_2 ( \mathbb{Z}_2)$이다.

\subsection{Case 3. $A$에 1이 세개}
$$
\begin{bmatrix}
1&0  \\ 
1&1
\end{bmatrix} \in N \implies
\begin{bmatrix}
1&0  \\ 
0&0
\end{bmatrix} 
\begin{bmatrix}
1&0  \\ 
1&1
\end{bmatrix} =
\begin{bmatrix}
1&0  \\ 
0&0
\end{bmatrix} \in N
$$이므로, Case 1.에 의해 $N = M_2 ( \mathbb{Z}_2)$이다.

\subsection{Case 4. $A$에 1이 네개}
$$
\begin{bmatrix}
1&1 \\ 
1&1
\end{bmatrix} \in N \implies
\begin{bmatrix}
1&0  \\ 
0&0
\end{bmatrix} 
\begin{bmatrix}
1&1  \\ 
1&1
\end{bmatrix} =
\begin{bmatrix}
1&1  \\ 
0&0
\end{bmatrix} \in N
$$
$$
\begin{bmatrix}
1&1 \\ 
0&0
\end{bmatrix} \in N \implies
\begin{bmatrix}
1&1  \\ 
0&0
\end{bmatrix} 
\begin{bmatrix}
1&0  \\ 
0&0
\end{bmatrix} =
\begin{bmatrix}
1&0  \\ 
0&0
\end{bmatrix} \in N
$$

이므로, Case 1.에 의해 $N = M_2 ( \mathbb{Z}_2)$이다.


따라서, $N$이 비자명 아이디얼인 모든 경우에 대해
$$N = M_2 ( \mathbb{Z}_2)$$임을 알 수 있다.

따라서 행렬환 $M_2 ( \mathbb{Z}_2)$은 단순환이다. \qed

\end{document}



































